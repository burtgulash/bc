\documentclass[12pt,titlepage]{report}
\usepackage[utf8]{inputenc}
\usepackage[czech]{babel}
\usepackage{amsmath}
\usepackage{amsfonts}
\usepackage{amssymb}
\usepackage{graphicx}
\usepackage{microtype}
\usepackage[Sonny]{fncychap}

\usepackage{algorithmicx}
\usepackage{algorithm}
\usepackage{algpseudocode}
\usepackage{caption}

\usepackage{wrapfig} % testing

\usepackage[pdfborder=0 0 0]{hyperref}

\begin{document}
\shorthandoff{-}

\begin{titlepage}
\begin{center}
	\Large{Západočeská univerzita v Plzni} \\
	\Large{Fakulta aplikovaných věd} \\
	\Large{Katedra infromatiky a výpočetní techniky} \\
\mbox{} \\[1.6cm]
	\LARGE{{\bf Bakalářská práce}} \\
\mbox{} \\
	\Huge{{\bf Měření významnosti autorů v citační síti}} \\
\end{center}
\vfill
\begin{minipage}{.5\textwidth}
Plzeň, 2013
\end{minipage}
\begin{minipage}{.5\textwidth}
\hfill Tomáš Maršálek
\end{minipage}
\thispagestyle{empty}
\end{titlepage}

\section*{Abstrakt}
Prvky sociální sítě, které nemají žádné apriorní ohodnocení významnosti, jsou
různě významné pouze na základě vztahů s okolními prvky. V této práci byly
prozkoumány a implementovány známé metody centrality a bibliografické metody
měřící významnost prvků v sociální nebo citační síti. Výsledky aplikování metod
na volně dostupné citační databáze ukázaly vysokou podobnost jednotlivých metod
a rovněž shodu nejvýznamnějších autorů dle těchto metod se známými oceněními v
oblasti informatiky a informační vědy (Turing Award, Codd, ACM Fellows, ISI
Highly Cited). Bylo zjištěno, že některé implementované metody jsou i přes
použití nejrychlejších algoritmů výpočetně příliš náročné vzhledem k velikosti
citačních sítí, vzniklých z těchto citačních databází. Dále bylo empiricky
potvrzeno, že implementované metody měří významnost, byť může mít více
interpretací.

\tableofcontents

\chapter{Úvod}

\chapter{Sociální a citační sítě}

\section{Sociální sítě}
Myšlenka sociální sítě existovala dlouho předtím, než je pod tímto termínem
začali lidé rozpoznávat. Jedná se o komplexní struktury vztahů mezi členy
sociálních uspořádání na všech úrovních - od osobních až po mezinárodní vztahy
mezi organizacemi.

Nejčastěji se ale setkáme se sociální sítí jako strukturou tvořenou lidmi,
kteří jsou svázáni nějakým sociálním vztahem.  Nejčastěji, zejména v poslední
době s rozmachem populárních webových sociálních sítí (MySpace, Facebook, G+,
Lidé), jím bývá přátelství.

\section{Reprezentace sítí}
Abychom mohli pracovat s doposud abstraktním konceptem sítě, musíme být schopni
ji reprezentovat jako datovou strukturu, na níž poté provedeme jakoukoliv
analýzu.
V odvětví matematiky teorie grafů je síť (graf) dvojice množin uzlů $V$
(vrcholů) a spojení uzlů $E$ (hran) $G = (V, E)$.  Obecně můžeme uvažovat grafy
s hranami s orientací či bez orientace. V obou případech se stále jedná o
dvojici $(V, E)$, pouze pro orientovaný graf je množina hran množinou
uspořádaných dvojic oproti množině neuspořádaných dvojic u neorientovaného
grafu.

V definici grafu je množina hran $E$ soubor dvojic, které označují koncové uzly
hrany, neboli jejich spojení. Samotné spojení je jediná informace, kterou
množina hran nese. Chceme-li zaznamenat nějakou další informaci, která je
spojená se spojením dvou uzlů, namísto hrany jako dvojice koncových uzlů
nadefinujeme hranu jako n-tici, kde první dvě hodnoty jsou koncové uzly a zbylé
hodnoty nesou libovolnou informaci. Ve většině případů si vystačíme s jednou
dodatečnou informací a nazýváme ji váha hrany. Jiná možnost pro zavedení vah
hran je váhová funkce $f: E \mapsto \mathbb{R}$, kde $f(e) = w$ je ohodnocení
konkrétní hrany $e \in E$. V případě zavedení vah hovoříme o vážených sítích.

Při zavedení vah máme například možnost používat síť jako multigraf, tedy graf,
u kterého je povoleno více spojení mezi dvěma stejnými uzly. Počet stejných
hran pak pouze zaznamenáme celočíselnou hodnotou ve váze hrany.

Například síť world wide web tvořená webovými stránkami je příkladem
multigrafu, protože je povoleno z jedné stránky odkazovat na jinou na více
místech. Při analýze takových sítí využijeme právě vah hran a počet
hypertextových odkazů mezi dvěma stránkami zaznamenáme vyšším ohodnocením
hrany. V tomhle případě znamená vyšší váha silnější pouto mezi uzly.

Jiným případem může být například síť kde sledujeme města a dopravní spojení
mezi nimi. V tomhle případě nás může zajímat vzdálenost nebo časová náročnost
na dopravu mezi dvěma městy, které budou znamenat silnější pouto pokud budou
mít naopak menší váhu. Hledáme totiž nejkratší a nejrychlejší spojení.


Pro reprezentaci v paměti počítače se nejčastěji používají dva způsoby - matice sousednosti a graf pomocí spojových seznamů.
Hrany se uzlu v případě orientovaného grafu liší z pohledu jednoho uzlu. Pokud
hrana vychází z tohoto uzlu, nazveme ji výstupní hrana, v opačném případě se
bude jednat o vstupní hranu.


Matice sousednosti (adjacency matrix) je čtvercová matice $A$ o velikosti počtu
vrcholů grafu $|V|$, ve které prvek ${\bf A}_{uv}$ na řádku $u$ a sloupci $v$
určuje jestli existuje hrana od vrcholu $u$ do vrcholu $v$. Pokud je hodnota
${\bf A}_{uv}$ $1$, hrana existuje; pokud je hodnota $0$, pak hrana neexistuje
a pokud je hodnota $w$, pak hrana existuje s váhou $w$.

Jiným maticovým způsobem uchování grafu je incidenční matice ${\bf B}$.
Incidenční matice vyjadřuje vztah mezi vrcholy a hranami tak, že ${\bf B}_{ue}
= 1$, pokud vrchol $u$ je spojený s hranou $e$, a $0$ v opačném případě. V
orientovaném grafu rozlišujeme mezi počátečním uzlem ${\bf B}_{ue} = -1$ a
koncovým uzlem ${\bf B}_{ue} = 1$. Incidenční matice se pro výpočetní teorii
grafů často nepoužívá z důvodu paměťové náročnosti, která je pro většinu grafů
výrazně vyšší než u matice sousednosti ($\Theta(|V||E|)$ oproti $\Theta(|V|^2)$, kde množina hran dosahuje velikostí $O(|V|^2)$).

Nejčastěji ale používáme myšlenku sousednosti vrcholů, ale namísto reprezentace
maticí, která je ve většině případů řídká a zbytečně obsahuje velké množství
nul, použijeme reprezentaci řídké matice - řádek nahradíme seznamem vrcholů,
které v matici sousednosti mají nenulovou hodnotu. Tento způsob je známý jako
graf pomocí spojových seznamů (adjacency list representation of a graph).

\section{Analýza sociálních sítí}

\subsection{Bezškálové sítě}
Více než 40 let byly považovány všechny komplexní sítě za naprosto náhodné.
Paul Erdős a Alfréd Rényi v roce 1959 navrhli modelování komunikačních sítí a
sítí, které se vyskytují v přírodních vědách, spojením uzlů náhodnými hranami.
Tento jednoduchý způsob způsobí rozložení stupňů vrcholů podle Poissonova
rozdělení s charakteristickou křivkou připomínající zvon - většina uzlů má
zhruba stejný stupeň. V roce 1998 bylo na univerzitě v Notre Dame (Barabási a
kolegové) provedeno mapování sítě World Wide Web s očekáváním, že výsledkem
bude náhodná síť. Přestože byl zmapován pouze zlomek celé sítě, výsledkem bylo
přes všechna očekávání, zcela jiné rozdělení stupňů - mocninné. Přes $80\%$
uzlů mělo méně než čtyři spojení, ale méně než $0.01\%$ uzlů mělo více než
tisíc spojení.
Sítě, které se řídí mocninným rozdělením, nazvali Barabási a jeho kolegové
bezškálovými sítěmi (scale-free network). Rozpoznání tohoto jevu vedlo k
lepšímu porozumění šíření virů a epidemií nebo proč některé sítě fungují takřka
beze změny i přes poruchu většiny jejich uzlů.
Sociální a citační sítě se řadí do kategorie bezškálových sítí. Například autor
vědecké literatury, jehož dílo je v dané oblasti známé, má velkou šanci, že
bude citován dalšími autory, především těmi novými. Stejně tak osoba v sociální
síti s velkým počtem přátel má velikou šanci, že bude představen novým lidem a
rozšíří si tak svůj kruh přátel ještě více. Tomuto jevu v bezškálových sítí se
říká \uv{bohatší se stává bohatším}.

\section{Citační sítě}
Citační sítě jsou podobné sociálním sítím, pouze místo uzlů, které představují
osoby, v citační síti se jedná o publikace nebo autory těchto publikací.  Pokud
je uzlem publikace, pak hrany této sítě symbolizují citaci publikace jinou
publikací. V druhém případě uvažujeme síť, kde uzly reprezentují autory knih,
vědeckých článků, vědecké literatury a dalších publikací. Prvnímu typu říkáme
síť publikací, druhému síť autorů.

\subsection{Síť publikací}
Uvažujeme-li první případ, kde uzly reprezentují publikace a hrany přímo citace
mezi těmito publikacemi, jedná se o síť publikací. Tedy pokud publikace $A$
odkazuje na publikaci $B$, pak budou existovat stejnojmenné uzly $A$ a $B$ a
hrana mezi těmito uzly může mít dvě různé orientace podle svého uplatnění. Směr
od citující publikace k citované (v našem příkladě od $A$ do $B$) bude mít
hrana, kterou označíme jako výstupní pro uzel $A$ a vstupní pro uzel $B$.
Výstupní hrana laicky řečeno označuje vztah "cituji", kdežto vstupní hrana
znamená "jsem citován".

\subsection{Síť autorů}
Druhým případem citační sítě je síť autorů. Zde je uzel reprezentací autora a
hrany spojují autory mezi sebou. Ve většině případech máme k dispozici data ve
formátu, který přímo odpovídá síti publikací, tzn. pro jednu publikaci známe
seznam jejích autorů a odkazů na další publikace. Síť autorů lze získat
transformací sítě publikací tak, že každou hranu z původní sítě publikací
přiřadíme každému z autorů této publikace a duplikujeme ji pro každého z autorů
citované publikace. Celkově vznikne $nm$ nových hran, pokud odkazovaná
publikace obsahuje $n$ autorů a odkazující $m$ autorů. Stejně jako v síti
publikací, i zde uvažujeme dvě opačné orientace hrany se stejnou interpretací,
tedy "cituji" a "jsem citován". 

V síti autorů má pro naše účely smysl uvažovat ohodnocení hran. Existuje více
způsobů, jak přiřadit ohodnocení (váhy) jednotlivým hranám, ale nejjednodušším
způsobem, který je použitý i v implementaci knihovny, je prosté přiřazení počtu
publikací, jejichž autorem nebo spoluautorem je daný autor $A$, které odkazují
na publikace, jejichž autorem je autor $B$. Srozumitelnější popis poskytne
obrázek:
%% TODO figure váha hran.

Druhým způsobem ohodnocení hran, který rovněž využívá implementovaná knihovna
pro některé metody, je převrácená hodnota prvního způsobu ohodnocení. Důvodem
je přímá souvislost mezi vahou hrany a vzdáleností mezi uzly. V prvním případě,
kdy silnější pouto mezi autory vyjadřuje vyšší ohodnocení hrany, v druhém
případě je naopak nižší váha vyjádřením silnějšího vztahu, jelikož jsou si uzly
blíže. Tento způsob je používán pro algoritmy, které pracují na myšlence
nejkratších cest mezi uzly. 

\subsection{Vážené citační sítě}
Pro citační sítě můžeme uvažovat ohodnocení hran obojího typu. Například mezi
dvěma autory může být silnější vztah, pokud se citují ve více publikacích.
Pokud citační síť analyzujeme metodami, které jsou založené na myšlence hledání
nejkratších cest i v této síti, která nemá v podstatě žádný pojem vzdálenosti,
použijeme druhý typ ohodnocení - menší váha, silnější pouto.


\subsection{Orientované a neorientované sítě}
V případě sociálních sítí nejčastěji uvažujeme sítě bez orientace, protože
nejčastěji modelovaný vztah přítel-přítel je ekvivalentní z pohledu obou
koncových uzlů. Pro citační síť jsou na místě orientované hrany, protože vztahy
autor odkazujícího na jiného autora nebo publikace citující jinou publikaci
mají očividně jinou interpretaci z pohledu koncových uzlů. Buďto se jedná o
citovaného nebo citujícího autora či publikaci.




\chapter{Citační databáze}
Citační databáze poskytují možnost vyhledávání bibliografických citací. Většina
z dnešních citačních databází se zaměřuje na jeden obor. Full-textové databáze
poskytují kompletní text publikací, které indexují (cite
http://library.amnh.org/research-tools/citation-full-text-databases).

\subsection{DBLP}
DBLP \cite{DBLP} je webová bibliografická databáze hostovaná na univerzitě
Trier. Od 80.  let indexovala literaturu z oblasti databází a logického
programovaní, ale postupně se její zaměření zobecnilo a nyní je bibliografickou
databází obecně pro obor informatiky. V roce 2012 obsahovala více než 2,1
milionu článků. Metody implementované v této práci jsou aplikovány na verzi z
roku 2004. 

\subsubsection{Charakteristika}
% TODO tzn. v jakym roce je nejvíc publikací atd.
\subsubsection{Struktura sítě}
\subsubsection{Rozďělení vah}

\subsection{CiteSeer}
CiteSeer (nyní CiteSeer$^X$) \cite{citeseer} je považován za první
automatizovaný systém shromažďování publikací a autonomní indexace citací v
nich obsažených. Publikace jsou zejména z oboru informatiky a informační vědy.
V dnešní době obsahuje přes dva miliony dokumentů s téměř dvěma miliony autorů
a čtyřiceti miliony citací. Zde používáme verzi z roku 2005.
\section{Ocenění významných autorů}
% TODO kecy co to ty ceny jsou a k čemu jsou a proč je tu uvádim.
\subsection{ACM A.M. Turing Award}
ACM A.M. Turing Award je ocenění ročně udělované skupinou ACM (Association for
Computing Machinery) jedincům vybraným pro kontribuce technického ducha do
výpočetního světa.
\cite{turingaward}.

Turingova cena je brána jako nejvyšší vyznamenání v informatice a je lidově
nazývána Nobelovou cenou pro informatiku \cite[p.~317]{dasgupta}.

\subsection{ACM SIGMOD Edgar F. Codd Innovations Award}
ACM SIGMOD Edgar F. Codd Innovations Award je ohodnocení životního díla
skupinou ACM SIGMOD (Special Interest Group on Management of Data)  za
inovativní a vysoce ceněné kontribuce k rozvoji, porozumění a použití
databázových systémů a databází \cite{sigmodinnovations}.

\subsection{ACM Fellows}
\uv{The ACM Fellows Program} byl založen v roce 1993, aby našel a ocenil
vynikající členy ACM za jejich dílo v informatice a informační vědě a pro
jejich významné kontribuce pro účel ACM. Členové ACM Fellows slouží jako
význační kolegové, ke kterým ACM a jejich členové vzhlížejí jako k autoritám v
době rozvoje informačních technologií \cite{acmfellows}.

\subsection{ISI Highly Cited highlighted}
ISI Highly Cited je databáze často citovaných autorů v článcích posledního
desetiletí, které byly vydány institutem ISI (Institute for Scientific
Information). Ten v dnešní době spadá pod agenturu Thomson Reuters, na jejíchž
webových stránkách nalezneme seznam autorů ISI Highly Cited highlighted z let
2000 až 2008 napříč 21 vědeckými obory \cite{highlycited}.

\section{Analýza citačních sítí}
\subsection{Souvislost a komponenty grafu}
Pro neorientovaný graf je komponenta maximálně souvislý podgraf. Jinak řečeno
komponenta je podgraf takový, že všechny jeho vrcholy jsou spojeny nějakou
cestou. Komponentou ji i samotný vrchol.

Všechny komponenty grafu najdeme pomocí jednoduchých algoritmů prohledávání do
šířky nebo do hloubky. Spuštění prohledávání najde celou komponentu, ve které
se výchozí vrchol nachází. Spustíme-li prohledávání ze všech vrcholů, najdeme
všechny komponenty. 

Slabě souvislý orientovaný graf znamená, že neorientovaný graf, který by vznikl
nahrazením orientovaných hran neorientovanými (symetrizace grafu), by byl
souvislý.

Pro zachování vlastnosti souvislosti, že všechny vrcholy jsou spojené nějakou
cestou, pro orientovaný graf musíme uvažovat silně souvislý graf nebo podgraf.
Definice zůstává stejná jako u slabě spojitých komponent, ale protože hrany
nejsou oboustranné, mezi dvěma spojenými vrcholy ne vždy existuje cesta oběma
směry. U neorientovaného grafu můžeme souvislost vyjádřit tak, že pro každé dva
uzly $u$ a $v$ existuje cesta z $u$ do $v$. Protože jsou hrany symetrické, pak
automaticky existuje i cesta z $v$ do $u$. U orientovaného grafu musíme druhou
podmínku explicitně dodat: graf je silně souvislý, pokud pro každé dva vrcholy
$u$ a $v$ existuje cesta z $u$ do $v$ i z $v$ do $u$. 

Silně spojité komponenty nenajdeme pouhým prohledáním do šířky nebo do hloubky,
ale použijeme sofistikovanější algoritmy (Kosarajův, Tarjanův, ...), které ale
vycházejí z prohledávání do hloubky.

\subsection{Klika v grafu}
Klika (clique) grafu je úplný podgraf. To znamená, že všechny vrcholy kliky
jsou spojeny přímo hranou.

V sociologii slovo klika souvisí se skupinou lidí, kteří jsou na sebe vázáni
více než na jiné lidi v tomtéž prostředí. Klika je silněji spojená skupina lidí
než sociální kruh.

\section{Významnost uzlů}
Významnost autorů je jedním předmětem zájmu analýzy sociálních sítí. Kdybychom
se měli rozhodnout, kterého člena sítě zvolit jako vůdce nebo přes které členy
nejrychleji rozšíříme zprávu, koho bychom měli vybrat? 

Velký díl k zodpovězení otázky relativní významnosti prvků definovali Freeman
(1979), Bonacich (1972), jehož práce je spojena s Hubbellovo (1965) mírou
sociometrického statusu, Coleman (1973) se svou mírou síly a Burt (1982) a jeho
míra prestiže (cite Noah E. Friedkin, Theoretical Foundations for Centrality Measures). Významnost prvku bývá v sociální síti označována jako centralita
a metody pro zjištění centrality jsou známé jako míry centrality (centrality
measure). Původně byly vyvinuty v sociologickém kontextu pro analýzu sociálních
sítí, ale jejich princip lze snadno zobecnit na obecný graf, proto můžeme
využít těchto metod pro analýzu citačních nebo jiných komplexních sítí, které
nemají čistě sociologický význam.

\section{Míry centrality}
\subsection{Degree}
Stupeň je počet hran spojených s uzlem. Pro orientovaný graf můžeme uvažovat
vstupní (indegree) a výstupní stupeň (outdegree) vrcholu nebo obecný stupeň
(degree), tedy součet těchto dvou.  Vstupní stupeň se často označuje jako
$deg^-$ a výstupní jako $deg^+$. 

\begin{align}
\label{eq:degree}
{C_D}_{in}(u) &= deg^-(u)  = \displaystyle\sum\limits_{v \in V} {\bf A}_{uv} \\
{C_D}_{out}(u) &= deg^+(u) = \displaystyle\sum\limits_{v \in V} {\bf A}_{vu} \\
{C_D}(v) &= {C_D}_{in} + {C_D}_{out}
\end{align}

Kde ${\bf A}$ je matice sousednosti grafu. Pokud uvažujeme pouze vstupní
stupeň, vypočtená hodnota určuje významnost uzlu, kdežto výstupní stupeň
ukazuje jakousi společenskost či otevřenost uzlu. 

Degree centrality je výpočetně velmi jednoduchý způsob, jak změřit významnost
prvku v síti. Tato metoda je však příliš jednoduchá, protože do výpočtu hodnoty
centrality nezahrnuje uzly, které jsou od daného uzlu vzdálenější než jeden
skok. Tento fakt je známý problém a důvod pro zavedení dalších a složitějších
metod pro výpočet významnosti.



\subsection{Eigenvector}
Eigenvector centrality, také známá jako Gould's index of accessibility of a
Network (Linear Algebra with Applications: Alternate Edition by Gareth
Williams) nebo Bonacich's centrality, je míra vlivu vrcholu v grafu, která
doslova znamená \uv{Důležitý uzel má důležité sousedy} (cite An introduction to Centrality measures, Zweig, Iyengar, 2010). Hodnotu vlivu získáme z
vlastního vektoru $x$ matice sousednosti grafu:

\begin{align}
{\bf Ax} &= \lambda {\bf x}
\end{align}

${\bf A}$ je matice sousednosti, ${\bf x}$ je vlastní vektor
matice ${\bf A}$ a řešením této rovnice o více řešeních. Ke každému řešení
náleží vlastní číslo $\lambda$. Pro měření významnosti nás však zajímá pouze to
řešení, které má pouze nezáporné hodnoty. Podle Perron-Frobeniovy věty pro
každou nezápornou primitivní matici existuje právě jedno takové řešení, které
zároveň patří k největšímu vlastnímu číslu $\lambda$ \cite{langvillemeyer}.

Rovnici můžeme rozepsat z maticového tvaru do jednotlivých složek:

\begin{align} 
x_u &=  \frac{1}{\lambda} \displaystyle\sum_{v \in G} {\bf A}_{uv} x_v 
\end{align} 

Kde $x_u$ je prvek vlastního vektoru ${\bf x}$
náležící vrcholu $u$ a ${\bf A}_{uv}$ je prvek matice sousednosti ${\bf A}$,
který leží na řádku $u$ a sloupci $v$.

\begin{align} 
{x_u}_{i + 1} &=  \frac{1}{\lambda} \displaystyle\sum_{v \in G} {\bf A}_{uv}
{x_v}_{i}
\end{align} 

V tomhle rekurentním tvaru je vidět předpis pro iterační výpočet eigenvector
centrality. Algoritmus se nazývá mocninná metoda, která se používá pro řešení
problému vlastních čísel v numerické matematice. Výsledkem mocinné metody je
dominantní vlastní číslo a odpovídající vlastní vektor. Pro eigenvector
centrality nás zajímá právě tohle řešení a žádné jiné.

Z druhé rovnice si navíc povšimneme, že se jedná o přímé rozšíření degree
centrality (\ref{eq:degree}). Výsledek předchozí iterace použijeme jako vstup
do následující a iterujeme tak dlouho, dokud nedosáhneme požadované přesnosti.

\subsubsection{PageRank}
V roce 1998 vyvinuli Sergey Brin a Larry Page algoritmus PageRank (nesoucí
jméno druhého autora) jako součást výzkumu na novém druhu webového vyhledávače
(cite něco). PageRank přiřazuje relativní hodnocení webovým stránkám podle
hypertextových odkazů z jiných webových stránek, které na ně směřují, a podle
jejich PageRankové významnosti. Sama definice je rekurzivní a po nahlédnutí na
vzorec zjistíme, že se jedná o rozšířenou variantu algoritmu pro eigenvector
centrality.

\begin{align}
{x_u}_{i + 1} &= \frac{1 - d}{|V|} + d \displaystyle\sum_{v \in V} {\bf A}_{uv}
\frac{{x_v}_{i}}{deg^+ (v)}
\end{align}

${\bf A}$ je opět matice sousednosti, $V$ je množina vrcholů a $deg^+(v)$ je
výstupní stupeň vrcholu $v$.  V původní myšlence, kdy PageRank modeluje chování
náhodného surfaře, damping factor je pravděpodobnost, že daný surfař přestane
opakovaně klikat na odkazy, které najde na webové stránce, na kterou se dostal
z předchozího odkazu, a otevře zcela novou stránku, ze které povede podobný
sled surfování přes odkazy.  Damping factor je často ze zkušenosti nastaven na
$85\%$.

Hodnota PageRanku je z matematického hlediska pravděpodobnost, že surfař, který
náhodně kliká na odkazy, se dostane na konkrétní stránku. Součet všech hodnot
PageRanku je tedy $1$, protože PageRank je rozdělení pravděpodobnosti.

Jedním problémem algoritmu PageRank jsou uzly bez výstupních hran (dangling
nodes). Protože musíme v každé iteraci algoritmu zachovat vlastnost rozdělení
pravděpodobnosti, že suma všech pravděpodobností je $1$, je třeba zajistit, aby
se přenášená hodnota mezi iteracemi neztrácela právě v uzlech bez výstupních
hran. Problém se nazývá rank sink a nejčastěji se řeší přidáním zdroje
PageRanku:

\begin{align}
{x_u}_{i + 1} &= \frac{1 - d}{|V|} + d \displaystyle\sum_{v \in V} {\bf A}_{uv} \frac{{x_v}_i}{deg^+ (v)} + \frac{1}{|V|} \displaystyle\sum_{w \in D} {x_w}_i
\end{align}

V každé iteraci předem vypočítáme součet hodnot PageRanku, které by se ztratily
v uzlech bez výstupních hran ($D$). Tahle hodnota je v rámci iterace konstantní
a pouze ji rovnoměrně rozdělíme mezi uzly sítě (s váhou $1/|V|$).

\begin{center}
\begin{minipage}{\textwidth}
\begin{algorithm}[H]
	\caption{PageRank}
		\label{alg:pagerank}

	\begin{algorithmic}[1]
	\Statex
	\State $x_0[v] \gets \frac{1}{|V|}$, $v \in V$\Comment{Uzly začínají se stejnou pravděpodobností}
	\For{$i \gets 0, K$}\Comment{Iterujeme pro $K$-krát}
		\State $s \gets 0$\Comment{$s$ je suma PageRanků pro dangling nodes}
		\For{$v \in V$}
			\If{$deg^+(v) = 0$}
				\State $s \gets s + x_{i}[v]$
			\EndIf
		\EndFor
		\State $x_{i + 1}[v] \gets \frac{s}{|V|}$, $v \in V$\Comment{Každý uzel začíná s přebytkem z dangling nodes}
		\For{$v \in V$}
			\For{$k \in inLinks[v]$}
				\State $x_{i + 1}[v] \gets x_{i + 1} + \frac{x_i[k]}{deg^+(k)}$
			\EndFor
		\EndFor
		\State $x_{i + 1}[v] \gets \frac{1 - d}{|V|} + d x_{i + 1}[v]$, $v \in V$\Comment{Přidání damping factoru}
	\EndFor
	\end{algorithmic}
\end{algorithm}
\end{minipage}
\end{center}
\mbox{}

Přestože je PageRank původně určený pro webovou síť, lze ho použít na
jakoukoliv orientovanou váženou i neváženou síť, tedy i na sociální a citační
sítě, o kterých je zde řeč. Pro neorientovaný graf je hodnota PageRanku pro
jednotlivé uzly velmi blízká stupňům grafu, ale ne totožná (cite icola Perra
and Santo Fortunato.; Fortunato (September 2008). "Spectral centrality measures
in complex networks")

\subsection{Míry založené na nejkratších cestách}
V sítích dopravní infrastruktury nás zajímá, po které cestě se nejrychleji a
nejvýhodněji dostat z bodu $A$ do bodu $B$. V sociálních a citačních sítích
nemůžeme intuitivně hovořit o nějakých cestách mezi uzly, protože ani přesně
nevíme jak takovou cestu interpretovat. Nejkratší cesta mezi přáteli v sociální
síti může znamenat, přes které přátele se mezi nimi nejpravděpodobněji šíří
informace. V sítích spolupráce vědeckých autorů se například setkáme s tzv.
Erdősovým číslem, které vyjadřuje nejkratší vzdálenost mezi osobou a
matematikem Paulem Erdősem v rámci spolupráce na matematických pracích.

Použijeme-li metody z dopravních sítí pro analýzu sociálních a citačních sítí,
které v jádře spočívají v hledání nejkratších cest, setkáme se se dvěma
nejznámějšími mírami centrality closeness a betweenness.

Nechť cesta z bodu $u \in V$ do bodu $v \in V$ je střídající se posloupnost
vrcholů a hran takových, že spojují předcházející a následující vrchol v této
posloupnosti. Délka cesty je pak součet vah hran této cesty nebo pouze počet
hran v případě neváženého grafu. Vzdálenost vrcholů $d_G(u, v)$ je délka
nejkratší z cest, které spojují vrcholy $u$ a $v$.

\subsection{Closeness}
Closeness neboli blízkost je definována jako převrácená hodnota míry farness,
tedy dalekost. Dalekost je součet všech vzdáleností od uzlu do všech ostatních,
tzn. $f(u) = \sum_{v \in V} d_G(u, v)$ a $c(u) = \sum_{v \in V} \frac{1}{d_G(u,
v)}$. Podle jiné definice je closeness převrácená hodnota průměrné nejkratší
cesty. V podstatě se od předchozí příliš neliší, protože průměrná nejkratší
cesta je rovna $\frac{1}{n
- 1} \sum_{v \in V} d_G(u, v)$ a closeness podle této definice:

\begin{align*}
c(u) &= \frac{n - 1}{\sum_{v \in V} d_G(u, v)}
\end{align*}

Pro obě definice platí, že čím vyšší hodnota $c(u)$, tím je uzel $u$
významnější podle této míry. Zde se budeme soustředit na druhou definici,
protože je častou volbou autorů zabývajících se touto problematikou a existuje
pro ni aproximační algoritmus, který si zde uvedeme.

Closeness, stejně jako ostatní míry centrality, modelují rozptýlení informace
napříč sítí. Výše uvedené klasické definici je vytýkáno, že pro přenos
informace uvažuje pouze nejkratší cesty, které nejsou vždy jedinou komunikační
cestou v síti. Alternativu navrhli Noh A Rieger (2004), kde namísto nejkratších
cest používají náhodné procházky (random walk closeness centrality). Příkladem
může být oběh mincí mezi lidmi. Tento jev nemá s nejkratšími cestami mnoho
společného, proto je vhodnější ho modelovat náhodnými procházkami. Oproti tomu
například poštovní zásilky očividně cestují po nejkratších cestách.  Pokud
uvažujeme citační sítě, nemáme jasnou představu o významu náhodných procházek
nebo nejkratších cest jako v případě mince nebo dopisu. I přesto očekáváme
vysokou podobnost této metody s ostatními.

Nevýhodou closeness centrality je nutnost uvažovat souvislý graf, tedy takový,
který obsahuje pouze jednu komponentu. Pokud by měl více komponent, pak by vždy
existovala cesta s nekonečnou vzdáleností. Hodnota farness by pak byla
automaticky nekonečná a closeness, tedy převrácená hodnota, by byla nulová. 

Existuje několik upravených definic, které se mají vypořádat s problémem
konektivity a druhotně jsou numericky stabilnější. Jedna z nich zaměňuje
převrácenou hodnotu součtu vzdáleností za součet převrácených hodnot
vzdáleností $c(u) = \sum_{v \in V} \frac{1}{d_G(u, v)}$ (Opsahl) a druhá $c(u)
= \sum_{v \in V} 2^{-d_G(u, v)}$ (Dangalchev). Přesto se nejvíce používá
původní definice closeness a výpočet se omezí na největší komponentu.



% TODO zmínit největší komponentu
\subsubsection{Algoritmus}
Closeness pro všechny vrcholy můžeme přesně vypočítat v čase $O(|V||E| +
|V|^2\log|V|)$, kde $V$ a $E$ jsou množiny vrcholů a hran sítě (cite JO77,
FT87). 

Algoritmus vychází z definice, tedy vyřeší problém všech párů nejkratších cest,
čímž rovnou získá hodnoty farness $f(u) = \sum_{v \in V} d_G(u, v)$ a zjištění
closeness je poté triviální podle jedné z výše uvedených definic.  Výše uvedená
složitost platí pro použití Dijkstrova algoritmu pro všechny páry cest.

Pro rozsáhlé sítě s miliony uzlů (sociální sítě k dnešnímu datu) je tato
metoda příliš náročná. Eppstein a Wang vyvinuli aproximační algoritmus s
náročností $O(\frac{\log|V|}{\epsilon}^2 (|V| \log |V| + |E|))$ s chybou
$\epsilon \delta$ pro převrácenou hodnotu closeness s pravděpodobností alespoň
$1 - \frac{1}{n}$, kde $\epsilon > 0$ a $\delta$ je diametr sítě (nejdelší z
nejkratších cest). Na základě tohoto aproximačního algoritmu byl vytvořen jiný
aproximační algoritmus pro nalezení $k$ nejvýznamnějších uzlů hodnocených podle
closeness centrality.

\begin{center}
\begin{minipage}{\textwidth}
\begin{algorithm}[H]
	\caption{Closeness}
		\label{alg:closeness}

	\begin{algorithmic}[1]
	\Statex
	\For{$s \in V$}
		\State $f \gets 0$\Comment{Farness}
		\State $d[v] \gets \infty$, $v \in V$\Comment{Zpočátku jsou uzly nedosažitelné}
		\State $d[u] \gets 0$
		\State $Q \gets \{s\}$\Comment{Prioritní fronta $Q$ začíná se zdrojovým vrcholem}
		\While{$Q \ne \varnothing$}\Comment{Dokud není fronta prázdná}
			\State $u \gets extract\mbox{-}min(Q)$\Comment{Vytáhneme uzel s min. vzdáleností $d$}
			\State $f \gets f + d[u]$\Comment{Zvýšíme farness pro nový uzel}
			\For{sousedící vrchol $v \in Adj[u]$}\Comment{Přidáme nově nalezené vrcholy}
				\If{$d[v] > d[u] + w(u, v)$}
					\State $d[v] \gets d[u] + w(u, v)$
					\State $Q \gets Q \cup \{v\}$
				\EndIf
			\EndFor
		\EndWhile
		\State $c[s] \gets \frac{n - 1}{f}$\Comment{Closeness je převrácená hodnota farness}
	\EndFor
	\end{algorithmic}
\end{algorithm}
\end{minipage}
\end{center}
\mbox{}

Výpočet closeness lze snadno paralelizovat, jelikož výpočet stromu nejkratších
cest je nezávislá úloha pro každý z $|V|$ vrcholů.

\subsubsection{Aproximace}
Algoritmus TOPRANK (Okamoto, Chen, Li) najde prvních $k$ nejvýznamnějších uzlů
s vysokou pravděpodobností a pro každý z nich přesnou hodnotu closeness.
Algoritmus pracuje s myšlenkou, že zjistíme přibližné pořadí uzlů tak, že pro
jeden strom nejkratších cest nebudeme počítat se všemi koncovými uzly, ale jen
s dostatečně velkým vzorkem této množiny.  Přesné hodnoty closeness dosáhneme
použitím exaktního algoritmu, který použijeme jen na nejvýznamnější uzly
získané z prvního aproximovaného kroku. Klíčovou otázkou je kolik
nejvýznamnějších uzlů musíme uvažovat, aby se jednalo o dostatečně přesný
výsledek. Autoři algoritmu uvádějí tento algoritmus s heuristikou, která najde
přibližně místo, ve kterém je vhodné výpočet ukončit a považovat za dostatečně
přesný. Sami uvádějí, že tento algoritmus je pouze první krok k návrhu
efektivnějšího způsobu jak najít prvních $k$ nejvýznamnějších uzlů.

\subsection{Betweenness}
Betweenness je druhá metoda, která modeluje šíření informace sítí pomocí
nejkratších cest. Princip betweenness spočívá v zvýhodnění uzlů s pozicí, přes
kterou teče nejvíce informace. Pokud uzel $A$ komunikuje s uzlem $C$, pak
můžeme tvrdit, že uzel $B$, který leží mezi nimi, bude mít roli prostředníka.
Být tímto prostředníkem mezi více takovými uzly intuitivně napovídá, že takový
uzel bude centrální. \uv{Čím více lidí na mně závisí k vytvoření spojení s
jinými lidmi, tím mám větší moc} (cite introductino to social network methods).
Betweenness měří na kolika nejkratších cestách se uzel nachází. Více se setkáme
s definicí, kde do sumy zahrneme poměr cest, na kterých se uzel nachází, k
celkovému počtu cest mezi dvěma uzly (Freeman, 1977; Anthonisse, 1971, Brandes).

\begin{align*}
b(v) &= \displaystyle\sum\limits_{s \in V} \displaystyle\sum\limits_{t \in V \backslash s} \frac{\sigma_{st}(v)}{\sigma_{st}} \\
\end{align*}

$b(v)$ značí hodnotu betweenness centrality pro uzel $v$, V množinu uzlů,
$\sigma_{st}$ je počet nejkratších cest mezi uzly $s$ a $t$ a $\sigma_{st}(v)$
je počet nejkratších cest, které navíc procházejí uzlem $v$.

Normalizovaný betweenness je hodnota v intervalu od $0$ do $1$, kterou získáme
tak, že betweenness vydělíme celkovým počtem možných cest - ($(|V| - 1)(|V| -
2)$) pro orientované grafy a ($\frac{(|V| - 1)(|V| - 2)}{2}$) pro neorientované
grafy. Normalizované hodnoty metod centralit jsou nezávislé na velikosti grafu
(cite Douglas R. WHite, Stephen P. Borgatti - Betweenness centrality measures
for directed graphs)

\begin{align*}
b(v) &= \frac{1}{(|V| - 1)(|V| - 2)} \displaystyle\sum\limits_{s \in V} \displaystyle\sum\limits_{t \in V \backslash s} \frac{\sigma_{st}(v)}{\sigma_{st}} \\
\end{align*}

Vznik betweenness je připisován sociologovi Lintonu Freemanovi (Freeman 77) a byl původně definován pro neorientované grafy.

\subsubsection{Brandesův algoritmus}
Ve své práci Ulrik Brandes zmiňuje do té doby nejrychlejší algoritmus pro
výpočet betweenness centrality s časovou náročností $\theta(|V|^3)$ a
$\theta(|V|^2)$ paměťovými nároky. Tento způsob přistupuje k problému
nekratších cest způsobem all-pair shortest paths. Brandesův způsob využívá
algoritmů pro nalezení nejkratších cest z jednoho bodu, kde výsledný algoritmus
pracuje s paměťovou náročností $O(|V| + |E|)$ a běží v čase $O(|V||E|)$ nebo
$O(|V||E| + |V|^2 \log|V|)$ pro nevážený, respektive vážený graf. 

%% TODO napsat pár větama brandes alg., aby se dala pochopit aproximace.

Brandes ve své práci o algoritmu uvádí pseudokód pro nevážený graf, který je
následně snadné pozměnit pro vážený graf zaměněním obyčejné fronty za prioritní
frontu; kompletní důkaz správnosti algoritmu a porovnání standardního algoritmu
s tímto (TODO cite Brandes).



\begin{center}
\begin{minipage}{\textwidth}
\begin{algorithm}[H]
	\caption{Brandesův algoritmus}
		\label{alg:brandes}

	\begin{algorithmic}[1]
	\Statex
	\State $b[v] \gets 0$, $v \in V$
	\For{$s \in V$}
		\State $S \gets $prázdný zásobník
		\State $P[w] \gets $prázdný seznam, $w \in V$
		\State $\sigma[t] \gets 0$, $t \in V$
		\State $\sigma[s] \gets 1$
		\State $d[t] \gets -1$, $t \in V$
		\State $d[s] \gets 0$
		\State $Q \gets \{s\}$
		\While{$Q \ne \varnothing$}
			\State $v \gets dequeue(Q)$
			\State $push(S, v)$
			\For{sousedící vrchol $w \in Adj[v]$}
				\If{$d[w] < 0$}
					\State $enqueue(Q, w)$
					\State $d[w] \gets d[v] + 1$
				\EndIf
				\If{$d[w] = d[v] + 1$}\Comment{Nejkratší cesta do $w$ přes $v$?}
					\State $\sigma[w] \gets \sigma[w] + \sigma[v]$
					\State $push(P[w], v)$
				\EndIf
			\EndFor
		\EndWhile
		\State $\delta[v] \gets 0$,$v \in V$\Comment{$\delta$ je závislost uzlu $s$ na ostatních}
		\While{$S \ne \varnothing$} \Comment{$S$ vrátí vrcholy v pořadí s nezvyšující se vzdáleností od $s$}
			\State $w \gets pop(S)$
			\For{$v \in P[w]$}
				\State $\delta[v] \gets \delta[v] + \frac{\sigma[v]}{\sigma[w]}(1 + \delta[w])$
			\EndFor
			\If{$w \ne s$}\Comment{Betweenness je součet dependencí $\delta[w]$}
				\State $b[w] \gets b[w] + \delta[w]$
			\EndIf
		\EndWhile
	\EndFor
	\end{algorithmic}
\end{algorithm}
\end{minipage}
\end{center}
\mbox{}


% Závislost (dependency) vrcholu $s \in V$ na všech ostatních vrcholech je v
% tomto algoritmu (cite Brandes):
% \begin{align*}
% \delta_{s \bullet}(v) &= \displaystyle\sum\limits_{w:v \in P_s(w)} \frac{\sigma_{sv}}{\sigma_{sw}}(1 + \sigma_{s \bullet}(w))
% \end{align*}

Algoritmus lze paralelizovat stejně jako algoritmus pro closeness centralitu,
pokud zajistíme synchronizaci vláken při přístupu k hodnotám betweenness.
Narozdíl od closeness, kde výpočet pro jeden uzel ovlivňuje hodnotu closeness
pouze pro tento uzel, u betweenness výpočet vycházející z jednoho uzlu ovlivňuje
hodnotu betweenness potencionálně i pro všechny ostatní uzly.

\subsubsection{Aproximace}
I přes rychlejší Brandesův algoritmus je výpočet betweenness centrality příliš
náročný výpočet pro sítě reálného světa (např. biologické, dopravní nebo webové
sítě) a pokud nám jde více o relativní pořadí uzlů podle hodnoty betweenness než
o hodnotu samotnou, lze oželit přesný výpočet přibližným, který příliš nezmění
výsledné umístění v žebříčku nejvýznamnějších uzlů.

Bader, Kintali, Madduri, Mihail ukazují aproximační algoritmus pro betweenness a
odhadem chyby.  Myšlenkou je jednoduchá lineární extrapolace Brandesova
algoritmu, pokud do výpočtu zahrneme pouze náhodný vzorek namísto celé množiny
vrcholů. Nechť $k$ je velikost vzorku množiny vrcholů, se kterým počítáme, pak
extrapolovaná hodnota betweenness je $\frac{|V| S}{k}$, kde $S$ je vypočtená
přibližná hodnota.



\section{Hledání nejkratších cest}
Hledání nejkratších cest v grafu je historicky starý problém, jehož matematický
výzkum přišel relativně pozdě v porovnání s jinými problémy kombinatorické
optimalizace (nejmenší kostra grafu, přiřazovací a dopravní problém).
Pravděpodobně byl výzkum opožděn, protože se jedná o intuitivní a relativně
jednoduchý problém, ale jakmile se dostal do středu zájmu, bylo nezávisle na
sobě nalezeno několik řešících metod různými lidmi (Shimbel [1955], Ford
[1956], Dantzig [1958], Bellman [1958], Moore[1959], Dijkstra [1959]).(cite
schrijver alexander ON THE HISTORY OF THE SHORTEST PATH PROBLEM)

Z hlediska metod řešení můžeme uvažovat několik kategorií algoritmů - nalezení
všech párů nejkratších cest (all-pairs shortest paths problem), nalezení cesty
mezi počátečním a koncovým vrcholem (source-target) nebo nalezení stromu
nejkratších cest, máme-li zadán počáteční vrchol (single source shortest path
problem).


\subsection{Single source shortest path}
Pokud hledáme pouze jednu cestu mezi dvěma vrcholy (source-target), nemusíme
počítat celý strom nejkratších cest, ale můžeme zastavit výpočet při dosažení
požadovaného vrcholu. 

\paragraph{BFS}
Prohledávání do šířky z jednoho bodu (breadth first search) je algoritmus,
který najde nejkratší cesty z jednoho bodu do všech ostatních v případě
neváženého grafu v čase $O(E)$. Pro vážený graf by zjistil nejkratší cesty, kde
metrika vzálenosti by byla počet skoků mezi uzly.

\begin{center}
\begin{minipage}{\textwidth}
\begin{algorithm}[H]
	\caption{Prohledávání do šířky}
		\label{alg:bfs}
	\begin{algorithmic}[1]
	\Statex
	\Function{BFS}{$G, s$}
		\State $d[v] \gets \infty$,$v \in V$\Comment{Uzly jsou zpočátku nedosažitelné}
		\State $d[s] \gets 0$
		\State $p[v] \gets $ NIL,$v \in V$\Comment{A nemají implicitně žádné předchůdce}
		\State $Q \gets \{s\}$\Comment{Fronta začíná s počátečním vrcholem}
		\While{$Q \ne \varnothing$}\Comment{Dokud není prázdná}
			\State $u \gets dequeue(Q)$
			\For{sousedící vrchol $v \in Adj[u]$}
				\If{$d[v] = -1$}\Comment{Pro všechny nově objevené vrcholy}
					\State $d[v] \gets d[u] + 1$\Comment{Nově vypočtená vzdálensot}
					\State $Q \gets Q \cup \{v\}$\Comment{Přidáme nově objevené uzly do fronty}
					\State $p[v] \gets u$\Comment{Předchůdce uzlu $v$ je $u$}
				\EndIf
			\EndFor
		\EndWhile
		\State \textbf{return} $p$
	\EndFunction
	\end{algorithmic}
\end{algorithm}
\end{minipage}
\end{center}
\mbox{}

\paragraph{Bellman-Fordův algoritmus}
Bellman-Fordův algoritmus je aplikací dynamického programování na nejkratší
cesty z jednoho vrcholu do všech ostatních. Jeho využití najdeme zejména tam,
kde se mohou objevit záporné váhy hran. Časová náročnost je $O(|V||E|)$.

\paragraph{Dijkstrův algoritmus}
Dijkstrův algoritmus řeší single source shortest path problém pomocí greedy
(žravé) strategie. Časová náročnost běžné implementace $O((|E| + |V|)\log|V|)$
je lepší než u Bellman-Fordova algoritmu, ale Dijkstrova greedy strategie
funguje pouze pokud v grafu neexistují záporně ohodnocené hrany. Časová
náročnost závisí především na implementaci klíčové datové struktury prioritní
fronty, která v každém kroku vybírá uzel s nejmenší vzdáleností od zdrojového
vrcholu (greedy strategie). V případě použití obyčejného spojového seznamu jako
prioritní fronty získáme kvadratickou náročnost $O(|V|^2)$. Nejčastěji se
setkáme s prioritní frontou implementovanou pomocí binární haldy, která
zajišťuje čas běhu právě $O((|E| + |V|)\log|V|)$. Pomocí Fibonacciho haldy
získáme doposud nejrychlejší Dijkstrův algoritmus s časem běhu $O(|E| + |V|
\log|V|)$, ale pro běžné účely (grafy menší než miliony vrcholů) je
nejvýhodnější binární halda (cite CLRS, Fredman \& Tarjan). Pro nevážený graf
je Dijkstrův algoritmus ekvivalentní prohledávání do šířky.

Dijkstrův algoritmus je klíčovým pro všechny implementované metody založené na
nejkratších cestách, proto uvedeme pseudokód (cite CLRS):

\begin{center}
\begin{minipage}{\textwidth}
\begin{algorithm}[H]
	\caption{Dijkstrův algoritmus}
		\label{alg:dijkstra}
	\begin{algorithmic}[1]
	\Statex
	\Function{Dijkstra}{$G, w, s$}  \Comment{$s$ je zdrojový vrchol a $w$ je váhová funkce}
		\State $d[v] \gets \infty$,$v \in V$\Comment{Vrcholy jsou zpočátku nedosažitelné}
		\State $d[s] \gets 0$
		\State $p[v] \gets $ NIL,$v \in V$\Comment{Předchůdci vrcholů zpočátku neexistují}
		\State $Q \gets V$\Comment{Prioritní fronta $Q$ obsahuje všechny vrcholy}
		\While{$Q \ne \varnothing$}\Comment{Postupujeme, dokud není prázdná}
			\State $u \gets extract\mbox{-}min(Q)$\Comment{Vytáhneme z fronty vrchol s nejmenší hodnotou $d[u]$}
			\For{sousedící vrchol $v \in Adj[u]$}
				\If{$d[v] > d[u] + w(u, v)$}
					\State $d[v] \gets d[u] + w(u, v)$\Comment{Relaxace hrany $\{u, v\}$}
					\State $p[v] \gets u$
				\EndIf
			\EndFor
		\EndWhile
		\State \textbf{return} $p$\Comment{Strom předchůdců}
	\EndFunction
	\end{algorithmic}
\end{algorithm}
\end{minipage}
\end{center}
\mbox{}

Výsledkem je strom předchůdců, který reprezentuje strom nejkratších cest.
Princip Dijkstrova algoritmu se objevuje i u algoritmů pro closeness,
betweenness a BFS. V tomto případě do prioritní fronty prvotně zahrneme všechny
vrcholy. Jinou možností je přidávat je postupně, aby vybírání z prioritní
fronty ($extract\mbox{-}min(Q)$) bylo rychlejší. Tento způsob je použit v
algoritmu pro closeness.
% TODO reference algs


\subsection{All-pair shortest paths}
Do této kategorie spadají maticové metody, tj. graf je zadán jako matice
sousednosti nebo matice sousednost s váhami hran.


\paragraph{Shimbelova metoda}
[1955] používá upravené maticové násobení k získání $|V|$-té mocniny matice
sousednosti. Celková časová náročnost je $O|V|^4$, protože provedeme $|V|$
násobení čtvercové matice o složitosti $O|V|^3$. Shimbelovo upravené násobení
nahrazuje sčítání a násobení za minimum a sčítání:

\begin{align*}
x + y &\equiv \min(x, y) \\
xy &\equiv x + y
\end{align*}

\paragraph{Floyd-Warshallův algoritmus} snižuje časovou náročnost na $O|V|^3$
použitím dynamického programování. Graf je opět zadán jako vážená matice
sousednosti. Rekurentní vzorec dynamického programování pro tento algoritmus je:

\begin{align*}
d_0(u, v) &= {\bf A}_{uv} \\
d_{k + 1}(u, v) &= \min(d_k(u, v), d_k(u, k) + d_k(k, v)) \\
\end{align*}

Jednodušše zkoušíme, zda je kratší cesta mezi vrcholy $u$ a $v$, kterou již
známe, nebo jiná cesta za použití nějakého vrcholu $k$, který leží mezi nimi.
Výpočet provádíme pro všechny páry vrcholů pro všechny vrcholy $k$ ($|V|^2
|V|$).

\paragraph{Johnsonův algoritmus} nepatří mezi maticové metody, protože využívá
metod single source shortest path pro všechny vrcholy. V principu jednodušše
použijeme Dijkstrův algoritmus pro každý vrchol zvlášť, ale dovolujeme váhy
hran i záporné. V případě záporných hran je nutné provést transformaci vah
pomocí Bellman-Fordova algoritmu, která v grafu nepozmění nejkratší cesty. 
(cite CLRS)


\section{Ostatní používané míry významnosti autorů}
\subsection{H-index}
H-index je metoda pro měření produktivity a významu publikované vědecké práce.
Metoda byla navržena fyzikem Jorge E. Hirschem pro zjištování relativní
významnosti vědců publikujících v oboru teoretické fyziky. Původní Hirschova
definice zní:

\begin{quotation}
Vědec má index $h$ pokud $h$ z jeho $N_p$ publikací má každá alespoň $h$ citací
a žádná ze zbylých ($N_p - h$) publikací nemá více než $h$ citací.
\end{quotation}

Tato metoda je specifická pro citační sítě a nelze ji aplikovat na sociální
nebo obecnou komplexní síť.

Výpočet probíhá tak, že seřadíme autorovy publikace $P$ sestupně podle počtu
citací $c: P \mapsto \mathbb{N}$ a poté od začátku tohoto seřazeného seznamu
hledáme tu publikaci, jejíž počet citací je nižší než pořadové číslo v tomto
seznamu. Toto číslo je pak h-index autora.
\begin{center}
\begin{minipage}{\textwidth}
\begin{algorithm}[H]
	\caption{H-index}
		\label{alg:hindex}

	\begin{algorithmic}[1]
	\Statex
	\Function{H-index}{$P, c$}
		\State $sort(P, key \gets c)$\Comment{Seřadíme publikace sestupně podle počtu citací}
		\State $h \gets 0$
		\For{$p \in P$}
			\If{$c(p) < h$}\Comment{Bod, kde počet citací vyrovná $h$ je h-index}
				\State \textbf{break}
			\EndIf
			\State $h \gets h + 1$
		\EndFor
		\State \textbf{return} $h$
	\EndFunction
	\end{algorithmic}
\end{algorithm}
\end{minipage}
\end{center}
\mbox{}


\section{Porovnání výsledků}
Pro zjištění, jestli jsou výsledky implementovaných metod shodné s uvedenými
oceněními, použijeme metodu součtu pořadí oceněných autorů. Tzn. pro jedno
ocenění sečteme pořadí všech autorů, kteří byli oceněni touhle cenou. Tuhle
jednoduchou míru můžeme porovnat pouze mezi jednotlivými metodami pro jedno
ocenění, ale ne mezi různými oceněními pro jednu metodu. Jednodušše protože
například Turingova cena je ve výsledcích udělena pouze několika autorům, z
čehož plyne malý součet pořadí, kdežto velké množství autorů na prvních
pozicích je členy ACM Fellows, tím pádem velký součet pořadí oceněných.


Dále nás zajímá, zda-li jsou metody mezi sebou podobné či nikoliv. Pro
porovnání metod mezi sebou je použit Spearmanův koeficient pořadové korelace.

\subsection{Spearmanův koeficient pořadové korelace}
Spearmanův koeficient je klasický Pearsonův koeficient korelace, který je
aplikovaný na proměnné s pořadím (cite Myers, Jerome L.; Well, Arnold D.
(2003), Research Design and Statistical Analysis (2nd ed.), Lawrence Erlbaum,
pp. 508).

Koeficient korelace obecně dosahuje hodnot od $-1$ do $1$, přičemž hodnota $1$
znamená naprostou lineární závislost mezi porovnávanými proměnnými tak, že s
rostoucí první proměnnou roste i druhá proměnná. Hodnota $-1$ znamená rovněž
naprostou lineární závislost, ale při rostoucí jedné proměnné druhá proměnná
klesá. Hodnota $0$ znamená kompletní náhodnost či nezávislost mezi měřenými
proměnnými.



% \section{Reprezentace citační sítě}

% \subsection{Paralelní výpočty}


% \subsubsection{Charakteristika}


\chapter{Výsledky}
\section{Žebříčky významných autorů}
Viz příloha A \ref{chapter:zebricky}. Hodnoty u metody PageRank jsou
transformovány z intervalu $[0; 1]$ na $[0; |V|]$.
% FIXME chapter:zebricky
\section{Porovnání nejvýznamnějších autorů}
\section{Porovnání implementovaných metod}


\chapter{Diskuse}
\section{Podobnost výsledků jednotlivých metod}
\paragraph{DBLP}
\begin{tabular}{|c|c|c|c|c|c|c|c|c|c|}
\hline
pagerank & 0.520 & 0.930 & 0.173 & 0.830 & 0.134 & 0.850 & 0.140 & 0.923 & - \\ 
\hline
pagerank & 0.514 & 0.993 & 0.210 & 0.871 & 0.130 & 0.869 & 0.136 & - \\ \cline{1-9}
w outdegree & 0.258 & 0.127 & 0.568 & 0.179 & 0.999 & 0.193 & - \\ \cline{1-8}
w indegree & 0.596 & 0.878 & 0.217 & 0.989 & 0.186 & - \\ \cline{1-7}
outdegree & 0.253 & 0.121 & 0.567 & 0.173 & - \\ \cline{1-6}
indegree & 0.571 & 0.880 & 0.203 & - \\ \cline{1-5}
radius & 0.300 & 0.203 & - \\ \cline{1-4}
betweenness & 0.528 & - \\ \cline{1-3}
h-index &  - \\ \cline{1-2}
 - \\ \cline{1-1}
\end{tabular}
\section{Shoda výsledků s oceněními}
\section{Vliv vah na přesnost výsledků}
\section{Vstupní a výstupní hrany}
\begin{tabular}{|c|c|c|c|c|c|c|c|c|c|}
\hline
   & WB & B & A & B & B & B & B & B & B \\
\hline
WB & 0.520 & 0.929 & 0.830 & 0.134 & 0.849 & 0.140 & 0.754 & 0.924 & - \\
\hline
B  & 0.514 & 0.993 & 0.872 & 0.133 & 0.869 & 0.138 & 0.809 & - \\
\cline{1-9}
A  & 0.533 & 0.818 & 0.909 & 0.102 & 0.898 & 0.108 & - \\
\cline{1-8}
B  & 0.258 & 0.126 & 0.178 & 0.998 & 0.192 & - \\
\cline{1-7}
B  & 0.596 & 0.877 & 0.988 & 0.186 & - \\
\cline{1-6}
B  & 0.252 & 0.121 & 0.173 & - \\
\cline{1-5}
B  & 0.571 & 0.880 & - \\
\cline{1-4}
B  & 0.517 & - \\
\cline{1-3}
B  &  - \\
\cline{1-2}
\end{tabular}

\begin{tabular}{|c|c|c|c|c|c|c|c|c|c|}
 - \\
0.51 & - \\
0.51 & 1.00 & - \\
0.51 & 1.00 & 1.00 & - \\
0.53 & 0.99 & 0.99 & 0.99 & - \\
0.53 & 0.99 & 0.99 & 0.99 & 1.00 & - \\
0.53 & 0.99 & 0.99 & 0.99 & 1.00 & 1.00 & - \\
0.53 & 0.57 & 0.57 & 0.57 & 0.57 & 0.57 & 0.57 & - \\
0.52 & 0.57 & 0.57 & 0.57 & 0.57 & 0.57 & 0.57 & 0.82 & - \\
0.52 & 0.57 & 0.57 & 0.57 & 0.58 & 0.58 & 0.58 & 0.81 & 0.99 & - \\
0.52 & 0.53 & 0.53 & 0.53 & 0.55 & 0.55 & 0.55 & 0.75 & 0.93 & 0.92 - \\
\end{tabular}

\chapter{Závěr}

\bibliographystyle{alpha}
\bibliography{references}

\newpage
\appendix
\chapter{Žebříčky významných autorů}
\label{chapter:zebricky}

\newpage
\section{DBLP}
\begin{footnotesize}
\input{../dblp/tabels/outdegree}
\input{../dblp/tabels/indegree}
\input{../dblp/tabels/wOutdegree}
\input{../dblp/tabels/wIndegree}
\input{../dblp/tabels/pagerank}
\input{../dblp/tabels/inCloseness}
\input{../dblp/tabels/wCloseness}
\input{../dblp/tabels/approximatedbetweennessnover4}
\input{../dblp/tabels/betweennessParallel}
\input{../dblp/tabels/wBetweenness}
\input{../dblp/tabels/hindex}
\end{footnotesize}

\section{CiteSeer}
\begin{footnotesize}
\end{footnotesize}

\end{document}
