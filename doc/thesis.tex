\documentclass[12pt,titlepage]{report}
\usepackage[utf8]{inputenc}
\usepackage[czech]{babel}
\usepackage{amsmath}
\usepackage{graphicx}
\usepackage{microtype}
\usepackage[Sonny]{fncychap}

\usepackage[pdfborder=0 0 0]{hyperref}

\begin{document}
\begin{titlepage}
\begin{center}
	\Large{Západočeská univerzita v Plzni} \\
	\Large{Fakulta aplikovaných věd} \\
	\Large{Katedra infromatiky a výpočetní techniky} \\
\mbox{} \\[1.6cm]
	\LARGE{{\bf Bakalářská práce}} \\
\mbox{} \\
	\Huge{{\bf Měření významnosti autorů v citační síti}} \\
\end{center}
\vfill
\begin{minipage}{.5\textwidth}
Plzeň, 2013
\end{minipage}
\begin{minipage}{.5\textwidth}
\hfill Tomáš Maršálek
\end{minipage}
\thispagestyle{empty}
\end{titlepage}

\section*{Abstrakt}
Prvky sociální sítě, které nemají žádné apriorní ohodnocení významnosti, jsou
různě významné pouze na základě vztahů s okolními prvky. V této práci byly
prozkoumány a implementovány známé metody centrality a bibliografické metody
měřící významnost prvků v sociální nebo citační síti. Výsledky aplikování metod
na volně dostupné citační databáze ukázaly vysokou podobnost jednotlivých metod
a rovněž shodu nejvýznamnějších autorů dle těchto metod se známými oceněními v
oblasti informatiky a informační vědy (Turing Award, Codd, ACM Fellows, ISI
Highly Cited). Bylo zjištěno, že některé implementované metody jsou i přes
použití nejrychlejších algoritmů výpočetně příliš náročné vzhledem k velikosti
citačních sítí, vzniklých z těchto citačních databází. Dále bylo empiricky
potvrzeno, že implementované metody měří významnost, byť může mít více
interpretací.

\tableofcontents

\chapter{Úvod}

\section{Sociální sítě}

\subsection{Analýza sociálních sítí}
\subsubsection{Bezškálové sítě}

\section{Citační sítě}
Citační sítě jsou podobné sociálním sítím, pouze místo uzlů, které představují
osoby, v citační síti se jedná o publikace nebo autory těchto publikací.  Pokud
je uzlem publikace, pak hrany této sítě symbolizují citaci publikace jinou
publikací. V druhém případě uvažujeme síť, kde uzly reprezentují autory knih,
vědeckých článků, vědecké literatury a dalších publikací. Prvnímu typu říkáme
síť publikací, druhému síť autorů.

\subsection{Síť publikací}
Uvažujeme-li první případ, kde uzly reprezentují publikace a hrany přímo citace
mezi těmito publikacemi, jedná se o síť publikací. Tedy pokud publikace $A$
odkazuje na publikaci $B$, pak budou existovat stejnojmenné uzly $A$ a $B$ a
hrana mezi těmito uzly může mít dvě různé orientace podle svého uplatnění. Směr
od citující publikace k citované (v našem příkladě od $A$ do $B$) bude mít
hrana, kterou označíme jako výstupní pro uzel $A$ a vstupní pro uzel $B$.
Výstupní hrana laicky řečeno označuje vztah "cituji", kdežto vstupní hrana
znamená "jsem citován".

\subsection{Síť autorů}
Druhým případem citační sítě je síť autorů. Zde je uzel reprezentací autora a
hrany spojují autory mezi sebou. Ve většině případech máme k dispozici data ve
formátu, který přímo odpovídá síti publikací, tzn. pro jednu publikaci známe
seznam jejích autorů a odkazů na další publikace. Síť autorů lze získat
transformací sítě publikací tak, že každou hranu z původní sítě publikací
přiřadíme každému z autorů této publikace a duplikujeme ji pro každého z autorů
citované publikace. Celkově vznikne $nm$ nových hran, pokud odkazovaná
publikace obsahuje $n$ autorů a odkazující $m$ autorů. Stejně jako v síti
publikací, i zde uvažujeme dvě opačné orientace hrany se stejnou interpretací,
tedy "cituji" a "jsem citován". 

V síti autorů má pro naše účely smysl uvažovat
ohodnocení hran. Existuje více způsobů, jak přiřadit ohodnocení (váhy)
jednotlivým hranám, ale nejjednodušším způsobem, který je použitý i v
implementaci knihovny, je prosté přiřazení počtu publikací, jejichž autorem
nebo spoluautorem je daný autor $A$, které odkazují na publikace, jejichž
autorem je autor $B$. Srozumitelnější popis poskytne obrázek:
%% TODO figure váha hran.

Druhým způsobem ohodnocení hran, který rovněž využívá implementovaná knihovna
pro některé metody, je převrácená hodnota prvního způsobu ohodnocení. Důvodem
je přímá souvislost mezi vahou hrany a vzdáleností mezi uzly. V prvním případě,
kdy silnější pouto mezi autory vyjadřuje vyšší ohodnocení hrany, v druhém
případě je naopak nižší váha vyjádřením silnějšího vztahu, jelikož jsou si uzly
blíže. Tento způsob je používán pro algoritmy, které pracují na myšlence
nejkratších cest mezi uzly. 

\subsection{Vážené citační sítě}
V definici grafu nebo sítě $G = (V, E)$ je množina hran $E$ soubor dvojic,
které označují koncové uzly hrany, neboli jejich spojení. Samotné spojení je
jediná informace, kterou množina hran nese. Chceme-li zaznamenat nějakou další
informaci, která je spojená se spojením dvou uzlů, namísto hrany jako dvojice
koncových uzlů nadefinujeme hranu jako n-tici, kde první dvě hodnoty jsou
koncové uzly a zbylé hodnoty nesou libovolnou informaci. Ve většině případů si
vystačíme s jednou dodatečnou informací a nazýváme ji váha hrany.

Při zavedení vah máme například možnost používat síť jako multigraf, tedy graf,
u kterého je povoleno více spojení mezi dvěma stejnými uzly. Počet stejných
hran pak pouze zaznamenáme celočíselnou hodnotou ve váze hrany.

Například síť world wide web tvořená webovými stránkami je příkladem
multigrafu, protože je povoleno z jedné stránky odkazovat na jinou na více
místech. Při analýze takových sítí využijeme právě vah hran a počet
hypertextových odkazů mezi dvěma stránkami zaznamenáme vyšším ohodnocením
hrany. V tomhle případě znamená vyšší váha silnější pouto mezi uzly.

Jiným případem může být například síť kde sledujeme města a dopravní spojení
mezi nimi. V tomhle případě nás může zajímat vzdálenost nebo časová náročnost
na dopravu mezi dvěma městy, které budou znamenat silnější pouto pokud budou
mít naopak menší váhu. Hledáme totiž nejkratší a nejrychlejší spojení.

Pro citační sítě můžeme uvažovat ohodnocení hran obojího typu. Například mezi
dvěma autory může být silnější vztah, pokud se citují ve více publikacích.
Pokud citační síť analyzujeme metodami, které jsou založené na myšlence hledání
nejkratších cest i v této síti, která nemá v podstatě žádný pojem vzdálenosti,
použijeme druhý typ ohodnocení - menší váha, silnější pouto.


\subsection{Orientované a neorientované sítě}
Obecně můžeme uvažovat grafy s hranami s orientací či bez orientace. V obou
případech se stále jedná o množinu $(V, E)$, pouze pro orientovaný graf je
množina hran množinou uspořádaných dvojic oproti množině neuspořádaných dvojic
u neorientovaného grafu.

Hrany se uzlu v případě orientovaného grafu liší z pohledu jednoho uzlu. Pokud
hrana vychází z tohoto uzlu, nazveme ji výstupní hrana, v opačném případě se
bude jednat o vstupní hranu.

V případě sociálních sítí nejčastěji uvažujeme sítě bez orientace, protože
nejčastěji modelovaný vztah přítel-přítel je ekvivalentní z pohledu obou
koncových uzlů. Pro citační síť jsou na místě orientované hrany, protože vztahy
autor odkazujícího na jiného autora nebo publikace citující jinou publikaci
mají očividně jinou interpretaci z pohledu koncových uzlů. Buďto se jedná o
citovaného nebo citujícího autora či publikaci.

\subsection{Souvislost a komponenty grafu}
Pro neorientovaný graf je komponenta maximálně souvislý podgraf. Jinak řečeno
komponenta je podgraf takový, že všechny jeho vrcholy jsou spojeny nějakou
cestou. Komponentou ji i samotný vrchol.

Všechny komponenty grafu najdeme pomocí jednoduchých algoritmů prohledávání do
šířky nebo do hloubky. Spuštění prohledávání najde celou komponentu, ve které
se výchozí vrchol nachází. Spustíme-li prohledávání ze všech vrcholů, najdeme
všechny komponenty. 

Slabě souvislý orientovaný graf znamená, že neorientovaný graf, který by vznikl
nahrazením orientovaných hran neorientovanými (symetrizace grafu), by byl
souvislý.

Pro zachování vlastnosti souvislosti, že všechny vrcholy jsou spojené nějakou
cestou, pro orientovaný graf musíme uvažovat silně souvislý graf nebo podgraf.
Definice zůstává stejná jako u slabě spojitých komponent, ale protože hrany
nejsou oboustranné, mezi dvěma spojenými vrcholy ne vždy existuje cesta oběma
směry. U neorientovaného grafu můžeme souvislost vyjádřit tak, že pro každé dva
uzly $u$ a $v$ existuje cesta z $u$ do $v$. Protože jsou hrany symetrické, pak
automaticky existuje i cesta z $v$ do $u$. U orientovaného grafu musíme druhou
podmínku explicitně dodat: graf je silně souvislý, pokud pro každé dva vrcholy
$u$ a $v$ existuje cesta z $u$ do $v$ i z $v$ do $u$. 

Silně spojité komponenty nenajdeme pouhým prohledáním do šířky nebo do hloubky,
ale použijeme sofistikovanější algoritmy (Kosarajův, Tarjanův, ...), které ale
vycházejí z prohledávání do hloubky.

\subsection{Klika v grafu}
Klika (clique) grafu je úplný podgraf. To znamená, že všechny vrcholy kliky
jsou spojeny přímo hranou.

V sociologii slovo klika souvisí se skupinou lidí, kteří jsou na sebe vázáni
více než na jiné lidi v tomtéž prostředí. Klika je silněji spojená skupina lidí
než sociální kruh.

\section{Analýza citačních sítí}
\section{Významnost autorů}
Ve světě
%% TODO tady se hodně zmínit o práci předchozích lidí, ne?

\section{Míry centrality}
\subsection{Degree}
Pro orientovaný graf můžeme uvažovat vstupní (indegree) a výstupní stupeň
(outdegree) vrcholu nebo obecný stupeň (degree), tedy součet těchto dvou.
Vstupní stupeň se často označuje jako $deg^-$ a výstupní jako $deg^+$.

\begin{align}
{C_D}_{in}(u) &= deg^-(u)  = \displaystyle\sum\limits_{v \in V} {\bf A}_{uv} \\
{C_D}_{out}(u) &= deg^+(u) = \displaystyle\sum\limits_{v \in V} {\bf A}_{vu} \\
{C_D}(v) &= {C_D}_{in} + {C_D}_{out}
\end{align}

${\bf A}$ je matice sousednosti grafu a prvek této matice ${\bf A}_{uv}$ na
řádku $u$ a $v$ značí, že existuje hrana z vrcholu $u$ do vrcholu $v$ - $1$
pokud hrana existuje, $w$ pokud se jedná o vážený graf a $0$, pokud hrana
neexistuje.

Pokud uvažujeme pouze vstupní stupeň, vypočtená hodnota určuje významnost uzlu, kdežto výstupní stupeň ukazuje jakousi společenskost či otevřenost uzlu. 

Degree centrality je výpočetně velmi jednoduchý způsob, jak změřit významnost
prvku v síti, v tomto případě autorů. Tato metoda je však příliš jednoduchá,
protože do výpočtu hodnoty centrality nezahrnuje uzly, které jsou od daného
uzlu vzdálenější než jeden skok. Tento fakt je známý problém a důvod pro
zavedení dalších a složitějších metod pro výpočet významnosti.



\subsection{Eigenvector}
Eigenvector centrality, také známá jako Gould's index of accessibility of a
Network (Linear Algebra with Applications: Alternate Edition by Gareth
Williams), je míra vlivu vrcholu v grafu. Hodnotu vlivu získáme z vlastního
vektoru $x$ matice sousednosti grafu:
\begin{align}
{\bf Ax} &= \lambda {\bf x} \\
\end{align}
${\bf A}$ je matice sousednosti (adjacency matrix), ${\bf x}$ je vlastní vektor
matice ${\bf A}$ a řešením této rovnice o více řešeních. Ke každému řešení
náleží vlastní číslo $\lambda$. Pro měření významnosti nás však zajímá pouze to
řešení, které má pouze nezáporné hodnoty. Podle Perron-Frobeniovy věty pro
každou nezápornou primitivní matici existuje právě jedno takové řešení, které
zároveň patří k největšímu vlastnímu číslu $\lambda$ \cite{langvillemeyer}.

Rovnici můžeme rozepsat z maticového tvaru do jednotlivých složek:

\begin{align} 
x_u &=  \frac{1}{\lambda} \displaystyle\sum_{v \in G} {\bf A}_{uv} x_v 
\end{align} 

Kde $x_u$ je prvek vlastního vektoru ${\bf x}$
náležící vrcholu $u$ a ${\bf A}_{uv}$ je prvek matice sousednosti ${\bf A}$,
který leží na řádku $u$ a sloupci $v$.

\begin{align} 
{x_u}_{i + 1} &=  \frac{1}{\lambda} \displaystyle\sum_{v \in G} {\bf A}_{uv}
{x_v}_{i}
\end{align} 

V tomhle rekurentním tvaru je vidět předpis pro iterační výpočet eigenvector
centrality. Algoritmus se nazývá mocninná metoda, která se používá pro řešení
problému vlastních čísel v numerické matematice. Výsledkem mocinné metody je
dominantní vlastní číslo a odpovídající vlastní vektor. Pro eigenvector
centrality nás zajímá právě tohle řešení a žádné jiné.

Z druhé rovnice si navíc povšimneme, že se jedná o přímé rozšíření degree
centrality (TODO referencuj tu rovnici u degree TODO). Výsledek předchozí
iterace použijeme jako vstup do následující iterace tak dlouho, dokud
nedosáhneme požadované přesnosti.

\subsubsection{PageRank}
V roce 1998 vyvinuli Sergey Brin a Larry Page algoritmus PageRank nesoucí jméno
druhého autora jako součást výzkumu na novém druhu webového vyhledávače.
PageRank přiřazuje relativní hodnocení webovým stránkám podle hypertextových
odkazů z jiných webových stránek, které na ně směřují, a podle jejich
PageRankové významnosti. Sama definice je rekurzivní a po nahlédnutí na vzorec
zjistíme, že se jedná pouze o rozšířenou variantu algoritmu pro eigenvector
centrality.

\begin{align}
{x_u}_{i + 1} &= \frac{1 - d}{|V|} + d \displaystyle\sum_{v \in V} {\bf A}_{uv}
\frac{{x_v}_{i}}{deg^+ (v)}
\end{align}

${\bf A}$ je opět matice sousednosti, $V$ je množina vrcholů a $deg^+(v)$ je
výstupní stupeň vrcholu $v$.  V původní myšlence, kdy PageRank modeluje chování
náhodného surfaře, damping factor je pravděpodobnost, že daný surfař přestane
opakovaně klikat na odkazy, které najde na webové stránce, na kterou se dostal
z předchozího odkazu, a otevře zcela novou stránku, ze které povede podobný
sled surfování přes odkazy.  Damping factor je často ze zkušenosti nastaven na
$85\%$.

Hodnota PageRanku je z matematického hlediska pravděpodobnost, že surfař, který
náhodně kliká na odkazy, se dostane na konkrétní stránku. Součet všech hodnot
PageRanku je tedy $1$, protože se vlastně jedná o rozdělení pravděpodobnosti.

Jedním problémem algoritmu PageRank jsou uzly bez výstupních hran (dangling
nodes). Protože musíme v každé iteraci algoritmu zachovat vlastnost rozdělení
pravděpodobnosti, že suma všech pravděpodobností je $1$, je třeba zajistit, aby
se přenášená hodnota mezi iteracemi neztrácela právě v uzlech bez výstupních
hran. Problém se nazývá rank sink a nejčastěji se řeší přidáním zdroje
PageRanku:

\begin{align}
{x_{i + 1}}_u &= \frac{1 - d}{|V|} + d \displaystyle\sum_{v \in V} {\bf A}_{uv} \frac{{x_i}_v}{deg^+ (v)} + \frac{1}{|D|} \displaystyle\sum_{w \in D} {x_i}_w
\end{align}

V každé iteraci předem vypočítáme součet hodnot PageRanku, které by se ztratily
v uzlech bez výstupních hran. Tahle hodnota je v rámci iterace konstantní a
pouze ji rovnoměrně rozdělíme mezi uzly sítě (tedy s váhou $1/|D|$, kde $D$ je
množina uzlů bez výstupních hran (dangling nodes).

\subsection{Míry založené na nejkratších cestách}
V sítích dopravní infrastruktury nás zajímá, po které cestě se nejrychleji a
nejvýhodněji dostat z bodu $A$ do bodu $B$. V sociálních a citačních sítích
nemůžeme intuitivně hovořit o nějakých cestách mezi uzly, protože ani přesně
nevíme jak takovou cestu interpretovat. Nejkratší cesta mezi přáteli v sociální
síti může znamenat, přes které přátele se mezi nimi nejpravděpodobněji šíří
informace. V sítích spolupráce vědeckých autorů se například setkáme s tzv.
Erdősovým číslem, které vyjadřuje nejkratší vzdálenost mezi osobou a
matematikem Paulem Erdősem v rámci spolupráce na matematických pracích.

Použijeme-li metody z dopravních sítí pro analýzu sociálních a citačních sítí,
které v jádře spočívají v hledání nejkratších cest, setkáme se s dvěma
nejznámějšími mírami centrality closeness a betweeness.

Nechť cesta z bodu $u \in V$ do bodu $v \in V$ je střídající se posloupnost
vrcholů a hran takových, že spojují předcházející a následující vrchol v této
posloupnosti. Délka cesty je pak součet vah hran této cesty nebo pouze počet
hran v případě neváženého grafu. Vzdálenost vrcholů $d_G(u, v)$ je délka
nejkratší z cest, které spojují vrcholy $u$ a $v$.

\subsection{Closeness}
Closeness neboli blízkost je definována jako převrácená hodnota míry farness,
tedy dalekost. Dalekost je součet všech vzdáleností od uzlu do všech ostatních,
tzn. $f(u) = \sum_{v \in V} d_G(u, v)$ a $c(u) = \sum_{v \in V} \frac{1}{d_G(u,
v)}$. Podle jiné definice je closeness převrácená hodnota průměrné nejkratší
cesty. V podstatě se od předchozí příliš neliší, protože průměrná nejkratší
cesta je rovna $\frac{1}{n
- 1} \sum_{v \in V} d_G(u, v)$ a closeness podle této definice:
\begin{align*}
c(u) &= \frac{n - 1}{\sum_{v \in V} d_G(u, v)}
\end{align*}

Pro obě definice platí, že čím vyšší hodnota $c(u)$, tím je uzel $u$
významnější podle této míry. Zde se budeme soustředit na druhou definici,
protože je častou volbou autorů zabývajících se touto problematikou a existuje
pro ni aproximační algoritmus, který si zde uvedeme.

Closeness, stejně jako ostatní míry centrality, modelují rozptýlení informace
napříč sítí. Výše uvedené klasické definici je vytýkáno, že pro přenos
informace uvažuje pouze nejkratší cesty, které nejsou vždy jedinou komunikační
cestou v síti. Alternativu navrhli Noh A Rieger (2004), kde namísto nejkratších
cest používají náhodné procházky (random walk closeness centrality). Příkladem
může být oběh mincí mezi lidmi. Tento jev nemá s nejkratšími cestami mnoho
společného, proto je vhodnější ho modelovat náhodnými procházkami. Oproti tomu
například poštovní zásilky očividně cestují po nejkratších cestách.  Pokud
uvažujeme citační sítě, nemáme jasnou představu o významu náhodných procházek
nebo nejkratších cest jako v případě mince nebo dopisu. I přesto očekáváme
vysokou podobnost této metody s ostatními.

Nevýhodou closeness centrality je nutnost uvažovat souvislý graf, tedy takový,
který obsahuje pouze jednu komponentu. Pokud by měl více komponent, pak by vždy
existovala cesta s nekonečnou vzdáleností. Hodnota farness by pak byla
automaticky nekonečná a closeness, tedy převrácená hodnota, by byla nulová.

Existuje několik upravených definic, které se mají vypořádat s problémem
konektivity a druhotně jsou numericky stabilnější. Jedna z nich zaměňuje
převrácenou hodnotu součtu vzdáleností za součet převrácených hodnot
vzdáleností $c(u) = \sum_{v \in V} \frac{1}{d_G(u, v)}$ (Opsahl) a druhá 
$c(u) = \sum_{v \in V} 2^{-d_G(u, v)}$ (Dangalchev).



% TODO zmínit největší komponentu
\subsubsection{Algoritmus}
Closeness pro všechny vrcholy můžeme přesně vypočítat v čase $O(|V||E| +
|V|^2\log|V|)$, kde $V$ a $E$ jsou množiny vrcholů a hran sítě (cite JO77,
FT87). 

Algoritmus vychází z definice, tedy vyřeší problém všech párů nejkratších cest,
čímž rovnou získá hodnoty farness $f(u) = \sum_{v \in V} d_G(u, v)$ a zjištění
closeness je poté triviální podle jedné z výše uvedených definic.  Výše uvedená
složitost platí pro použití Dijkstrova algoritmu (cite Dijkstra) pro všechny
páry cest.

Pro rozsáhlé sítě s miliony uzlů (sociální sítě k dnešnímu datu) je tato
metoda příliš náročná. Eppstein a Wang vyvinuli aproximační algoritmus s
náročností $O(\frac{\log|V|}{\epsilon}^2 (|V| \log |V| + |E|))$ s chybou
$\epsilon \delta$ pro převrácenou hodnotu closeness s pravděpodobností alespoň
$1 - \frac{1}{n}$, kde $\epsilon > 0$ a $\delta$ je diametr sítě (nejdelší z
nejkratších cest). Na základě tohoto aproximačního algoritmu byl vytvořen jiný
aproximační algoritmus pro nalezení $k$ nejvýznamnějších uzlů hodnocených podle
closeness centrality.

\subsubsection{Aproximace}
Algoritmus TOPRANK (Okamoto, Chen, Li) najde prvních $k$ nejvýznamnějších uzlů
s vysokou pravděpodobností a pro každý z nich přesnou hodnotu closeness.
Algoritmus pracuje s myšlenkou, že zjistíme přibližné pořadí uzlů tak, že pro
jeden strom nejkratších cest nebudeme počítat se všemi koncovými uzly, ale jen
s dostatečně velkým vzorkem této množiny.  Přesné hodnoty closeness dosáhneme
použitím exaktního algoritmu, který použijeme jen na nejvýznamnější uzly
získané z prvního aproximovaného kroku. Klíčovou otázkou je kolik
nejvýznamnějších uzlů musíme uvažovat, aby se jednalo o dostatečně přesný
výsledek. Autoři algoritmu uvádějí tento algoritmus s heuristikou, která najde
přibližně místo, ve kterém je vhodné výpočet ukončit a považovat za dostatečně
přesný. Sami uvádějí, že tento algoritmus je pouze první krok k návrhu
efektivnějšího způsobu jak najít prvních $k$ nejvýznamnějších uzlů.

\subsection{Betweeness}
Betweeness je druhá metoda, která modeluje šíření informace sítí pomocí
nejkratších cest. Princip betweeness spočívá v zvýhodnění uzlů s pozicí, přes
kterou teče nejvíce informace. Pokud uzel $A$ komunikuje s uzlem $C$, pak
můžeme tvrdit, že uzel $B$, který leží mezi nimi, bude mít roli prostředníka.
Být tímto prostředníkem mezi více takovými uzly intuitivně napovídá, že takový
uzel bude centrální. \uv{Čím více lidí na mně závisí k vytvoření spojení s
jinými lidmi, tím mám větší moc} (cite introductino to social network methods).
Betweeness měří na kolika nejkratších cestách se uzel nachází. Více se setkáme
s definicí, kde do sumy zahrneme poměr cest, na kterých se uzel nachází, k celkovému počtu cest mezi dvěma uzly.

\begin{align*}
b(v) &= \displaystyle\sum\limits_{s \in V} \displaystyle\sum\limits_{t \in V \backslash s} \frac{\sigma_{st}(v)}{\sigma_{st}} \\
\end{align*}

$b(v)$ značí hodnotu betweeness centrality pro uzel $v$, V množinu uzlů,
$\sigma_{st}$ je počet nejkratších cest mezi uzly $s$ a $t$ a $\sigma_{st}(v)$
je počet nejkratších cest, které navíc procházejí uzlem $v$.

Normalizovaný betweeness je hodnota v intervalu od $0$ do $1$, kterou získáme
tak, že betweeness vydělíme celkovým počtem možných cest - ($(|V| - 1)(|V| -
2)$) pro orientované grafy a ($\frac{(|V| - 1)(|V| - 2)}{2}$) pro neorientované
grafy.

\begin{align*}
b(v) &= \frac{1}{(|V| - 1)(|V| - 2)} \displaystyle\sum\limits_{s \in V} \displaystyle\sum\limits_{t \in V \backslash s} \frac{\sigma_{st}(v)}{\sigma_{st}} \\
\end{align*}

Vznik betweeness je připisován sociologovi Lintonu Freemanovi (Freeman 77).

\subsubsection{Brandesův algoritmus}
\subsubsection{Aproximace}
\subsection{Radius}

\section{Ostatní používané míry významnosti autorů}
\subsection{H-index}
\subsection{Impact factor}

\section{Nejkratší cesty}

\section{Porovnání výsledků}

\section{Ocenění významných autorů}
% TODO kecy co to ty ceny jsou a k čemu jsou a proč je tu uvádim.
\subsection{ACM A.M. Turing Award}
ACM A.M. Turing Award je ocenění ročně udělované skupinou ACM (Association for
Computing Machinery) jedincům vybraným pro kontribuce technického ducha do
výpočetního světa.
\cite{turingaward}.

Turingova cena je brána jako nejvyšší vyznamenání v informatice a je lidově
nazývána Nobelovou cenou pro informatiku \cite[p.~317]{dasgupta}.

\subsection{ACM SIGMOD Edgar F. Codd Innovations Award}
ACM SIGMOD Edgar F. Codd Innovations Award je ohodnocení životního díla
skupinou ACM SIGMOD (Special Interest Group on Management of Data)  za
inovativní a vysoce ceněné kontribuce k rozvoji, porozumění a použití
databázových systémů a databází \cite{sigmodinnovations}.

\subsection{ACM Fellows}
\uv{The ACM Fellows Program} byl založen v roce 1993, aby našel a ocenil
vynikající členy ACM za jejich dílo v informatice a informační vědě a pro
jejich významné kontribuce pro účel ACM. Členové ACM Fellows slouží jako
význační kolegové, ke kterým ACM a jejich členové vzhlížejí jako k autoritám v
době rozvoje informačních technologií \cite{acmfellows}.

\subsection{ISI Highly Cited highlighted}
ISI Highly Cited je databáze často citovaných autorů v článcích posledního
desetiletí, které byly vydány institutem ISI (Institute for Scientific
Information). Ten v dnešní době spadá pod agenturu Thomson Reuters, na jejíchž webových stránkách nalezneme seznam autorů ISI Highly Cited highlighted z let 2000 až 2008 napříč 21 vědeckými obory \cite{highlycited}.

\chapter{Implementace}
% \section{Načtení vstupních dat}

\section{Vytvoření citačních sítí}
\subsection{Síť publikací}
\subsection{Síť autorů}

\section{Reprezentace citační sítě}
\section{Analýza struktury sítě}

\section{Hledání nejkratších cest}
\subsection{All-pair shortest path}
\subsection{Single source shortest path}

\section{Knihovna pro SNA}
\subsection{Radius}
\subsection{Degree}
\subsection{Eigenvector}
\subsection{Closeness}
\subsection{Betweeness}
% TODO normalizace - vydělení maximální možnou hodnotou
\subsection{H-index}

\subsection{Paralelní výpočty}

\section{Citační databáze}
\subsection{DBLP}
DBLP \cite{DBLP} je webová bibliografická databáze v oboru informatiky,
která k listopadu 2012 obsahovala více než 2,1 milionu publikací. Pro tuto
práci používáme verzi z roku 2004.

\subsubsection{Charakteristika}
% TODO tzn. v jakym roce je nejvíc publikací atd.
\subsubsection{Struktura sítě}
\subsubsection{Rozďělení vah}

\subsection{CiteSeer}
CiteSeer (nyní CiteSeer$^X$) \cite{citeseer} je považován za první
automatizovaný systém shromažďování publikací a autonomní indexace citací v
nich obsažených. Publikace jsou zejména z oboru informatiky a informační vědy.
V dnešní době obsahuje přes dva miliony dokumentů s téměř dvěma miliony autorů
a čtyřiceti miliony citací. Zde používáme verzi z roku 2005.

\subsubsection{Charakteristika}


\chapter{Výsledky}
\section{Porovnání nejvýznamnějších autorů}
\section{Porovnání implementovaných metod}

\section{Žebříčky významných autorů}
\subsection{DBLP}
\subsubsection{H-index}
\begin{center}
\begin{tabular}{|l|l|c|c|c|c|c|}
\hline
& {\bf Autor} & {\bf H-index} & {\bf Turing} & {\bf Codd} & {\bf Fellows} & {\bf ISI} \\
\hline
1 & MICHAEL STONEBRAKER & 28& & $\bullet$ & $\bullet$ &         \\
\hline
2 & DAVID J. DEWITT & 24& & $\bullet$ & $\bullet$ &         \\
\hline
3 & JEFFREY D. ULLMAN & 24& & $\bullet$ & $\bullet$ & $\bullet$ \\
\hline
4 & PHILIP A. BERNSTEIN & 22& & $\bullet$ & $\bullet$ & $\bullet$ \\
\hline
5 & RAKESH AGRAWAL & 21& & $\bullet$ & $\bullet$ &         \\
\hline
6 & WON KIM & 21& & $\bullet$ & $\bullet$ &         \\
\hline
7 & YEHOSHUA SAGIV & 20& &         &         &         \\
\hline
8 & CATRIEL BEERI & 20& &         & $\bullet$ & $\bullet$ \\
\hline
9 & MICHAEL J. CAREY & 20& & $\bullet$ & $\bullet$ &         \\
\hline
10 & SERGE ABITEBOUL & 19& & $\bullet$ & $\bullet$ & $\bullet$ \\
\hline
11 & HECTOR GARCIA-MOLINA & 19& & $\bullet$ & $\bullet$ & $\bullet$ \\
\hline
12 & UMESHWAR DAYAL & 19& & $\bullet$ & $\bullet$ &         \\
\hline
13 & CHRISTOS FALOUTSOS & 19& &         & $\bullet$ & $\bullet$ \\
\hline
14 & NATHAN GOODMAN & 18& & $\bullet$ &         &         \\
\hline
15 & JIM GRAY & 18& &         & $\bullet$ &         \\
\hline
16 & JEFFREY F. NAUGHTON & 18& &         & $\bullet$ &         \\
\hline
17 & RAGHU RAMAKRISHNAN & 18& &         &         &         \\
\hline
18 & RONALD FAGIN & 18& & $\bullet$ & $\bullet$ & $\bullet$ \\
\hline
19 & JENNIFER WIDOM & 18& & $\bullet$ & $\bullet$ &         \\
\hline
20 & DAVID MAIER & 17& & $\bullet$ & $\bullet$ & $\bullet$ \\
\hline
21 & BRUCE G. LINDSAY & 17& &         & $\bullet$ &         \\
\hline
22 & SHAMKANT B. NAVATHE & 16& &         &         &         \\
\hline
23 & C. MOHAN & 16& & $\bullet$ & $\bullet$ &         \\
\hline
24 & HAMID PIRAHESH & 16& &         & $\bullet$ &         \\
\hline
25 & H. V. JAGADISH & 16& &         & $\bullet$ &         \\
\hline
\end{tabular}
\end{center}

\subsubsection{Nevážený indegree}
\begin{center}
\begin{tabular}{|l|l|c|c|c|c|c|}
\hline
& {\bf Autor} & {\bf indegree} & {\bf Turing} & {\bf Codd} & {\bf Fellows} & {\bf ISI} \\
\hline
1  & MICHAEL STONEBRAKER & 1909&         &$\bullet$&$\bullet$& \\
\hline
2  & DAVID J. DEWITT & 1484&         &$\bullet$&$\bullet$& \\
\hline
3  & JIM GRAY & 1400&         &$\bullet$&$\bullet$& \\
\hline
4  & RAYMOND A. LORIE & 1276&         &         &$\bullet$& \\
\hline
5  & JEFFREY D. ULLMAN & 1180&         &$\bullet$&$\bullet$&$\bullet$ \\
\hline
6  & WON KIM & 1146&         &         &$\bullet$& \\
\hline
7  & PHILIP A. BERNSTEIN & 1145&         &$\bullet$&$\bullet$&$\bullet$ \\
\hline
8  & E. F. CODD & 1110&$\bullet$&         &$\bullet$& \\
\hline
9  & MICHAEL J. CAREY & 1110&         &$\bullet$&$\bullet$& \\
\hline
10 & UMESHWAR DAYAL & 1076&         &$\bullet$&$\bullet$& \\
\hline
11 & HECTOR GARCIA-MOLINA & 1020&         &$\bullet$&$\bullet$&$\bullet$ \\
\hline
12 & DAVID MAIER & 1017&         &$\bullet$&$\bullet$&$\bullet$ \\
\hline
13 & DONALD D. CHAMBERLIN & 966&         &$\bullet$&$\bullet$& \\
\hline
14 & RAKESH AGRAWAL & 907&         &$\bullet$&$\bullet$& \\
\hline
15 & PETER P. CHEN & 906&         &         &$\bullet$& \\
\hline
16 & SERGE ABITEBOUL & 848&         &$\bullet$&$\bullet$&$\bullet$ \\
\hline
17 & KAPALI P. ESWARAN & 847&         &         &         & \\
\hline
18 & MORTON M. ASTRAHAN & 846&         &         &         & \\
\hline
19 & FRANCCEDILOIS BANCILHON & 840&         &         &         & \\
\hline
20 & NATHAN GOODMAN & 819&         &         &         &$\bullet$ \\
\hline
21 & BRUCE G. LINDSAY & 806&         &         &$\bullet$& \\
\hline
22 & HAMID PIRAHESH & 803&         &         &$\bullet$& \\
\hline
23 & IRVING L. TRAIGER & 785&         &         &$\bullet$& \\
\hline
24 & EUGENE WONG & 762&         &         &         & \\
\hline
25 & JEFFREY F. NAUGHTON & 729&         &         &$\bullet$& \\
\hline
\end{tabular}
\end{center}

\subsubsection{Nevážený outdegree}
\begin{center}
\begin{tabular}{|l|l|c|c|c|c|c|}
\hline
& {\bf Autor} & {\bf indegree} & {\bf Turing} & {\bf Codd} & {\bf Fellows} & {\bf ISI} \\
\hline
1  & GERHARD WEIKUM & 872&&         &$\bullet$&         \\
\hline
2  & HECTOR GARCIA-MOLINA & 856&&$\bullet$&$\bullet$&$\bullet$\\
\hline
3  & RAKESH AGRAWAL & 761&&$\bullet$&$\bullet$&         \\
\hline
4  & MICHAEL J. CAREY & 758&&$\bullet$&$\bullet$&         \\
\hline
5  & DAVID J. DEWITT & 758&&$\bullet$&$\bullet$&         \\
\hline
6  & H. V. JAGADISH & 717&&         &$\bullet$&         \\
\hline
7  & MICHAEL STONEBRAKER & 677&&$\bullet$&$\bullet$&         \\
\hline
8  & RAGHU RAMAKRISHNAN & 652&&         &$\bullet$&         \\
\hline
9  & YANNIS E. IOANNIDIS & 649&&         &$\bullet$&         \\
\hline
10 & ABRAHAM SILBERSCHATZ & 636&&         &$\bullet$&         \\
\hline
11 & ELISA BERTINO & 635&&         &$\bullet$&         \\
\hline
12 & SHAMKANT B. NAVATHE & 629&&         &         &         \\
\hline
13 & PHILIP S. YU & 622&&         &$\bullet$&$\bullet$\\
\hline
14 & STEFANO CERI & 611&&         &         &         \\
\hline
15 & CHRISTOS FALOUTSOS & 607&&         &$\bullet$&         \\
\hline
16 & MATTHIAS JARKE & 586&&         &         &         \\
\hline
17 & GULTEKIN OUMLZSOYOGLU & 582&&         &         &         \\
\hline
18 & SERGE ABITEBOUL & 575&&$\bullet$&$\bullet$&$\bullet$\\
\hline
19 & NICK ROUSSOPOULOS & 568&&         &$\bullet$&         \\
\hline
20 & MIRON LIVNY & 559&&         &         &         \\
\hline
21 & STANLEY Y. W. SU & 558&&         &         &         \\
\hline
22 & HANS-JOUMLRG SCHEK & 557&&         &$\bullet$&         \\
\hline
23 & PATRICK VALDURIEZ & 547&&         &$\bullet$&         \\
\hline
24 & GOETZ GRAEFE & 546&&         &         &         \\
\hline
25 & CLEMENT T. YU & 542&&         &         &         \\
\hline
\end{tabular}
\end{center}

\subsubsection{Vážený indegree}
\begin{center}
\begin{tabular}{|l|l|c|c|c|c|c|}
\hline
& {\bf Autor} & {\bf indegree} & {\bf Turing} & {\bf Codd} & {\bf Fellows} & {\bf ISI} \\
\hline
1  & MICHAEL STONEBRAKER & 5946    &         &$\bullet$&$\bullet$&         \\
\hline
2  & DAVID J. DEWITT & 5733        &         &$\bullet$&$\bullet$&         \\
\hline
3  & JEFFREY D. ULLMAN & 4429      &         &$\bullet$&$\bullet$&$\bullet$\\
\hline
4  & JIM GRAY & 3982               &         &$\bullet$&$\bullet$&         \\
\hline
5  & MICHAEL J. CAREY & 3583       &         &$\bullet$&$\bullet$&         \\
\hline
6  & RAYMOND A. LORIE & 3501       &         &         &$\bullet$&         \\
\hline
7  & HECTOR GARCIA-MOLINA & 3275   &         &$\bullet$&$\bullet$&$\bullet$\\
\hline
8  & PHILIP A. BERNSTEIN & 3225    &         &$\bullet$&$\bullet$&$\bullet$\\
\hline
9  & SERGE ABITEBOUL & 3177        &         &$\bullet$&$\bullet$&$\bullet$\\
\hline
10 & RAKESH AGRAWAL & 3152         &         &$\bullet$&$\bullet$&         \\
\hline
11 & WON KIM & 2993                &         &         &$\bullet$&         \\
\hline
12 & DAVID MAIER & 2772            &         &$\bullet$&$\bullet$&$\bullet$\\
\hline
13 & E. F. CODD & 2736             &$\bullet$&         &$\bullet$&         \\
\hline
14 & YEHOSHUA SAGIV & 2575         &         &         &         &         \\
\hline
15 & CATRIEL BEERI & 2491          &         &         &$\bullet$&$\bullet$\\
\hline
16 & UMESHWAR DAYAL & 2465         &         &$\bullet$&$\bullet$&         \\
\hline
17 & RAGHU RAMAKRISHNAN & 2426     &         &         &$\bullet$&         \\
\hline
18 & CHRISTOS FALOUTSOS & 2413     &         &         &$\bullet$&         \\
\hline
19 & JENNIFER WIDOM & 2354         &         &$\bullet$&$\bullet$&         \\
\hline
20 & DONALD D. CHAMBERLIN & 2269   &         &$\bullet$&$\bullet$&         \\
\hline
21 & FRANCCEDILOIS BANCILHON & 2264&         &         &         &         \\
\hline
22 & JEFFREY F. NAUGHTON & 2186    &         &         &$\bullet$&         \\
\hline
23 & NATHAN GOODMAN & 2176         &         &         &         &$\bullet$\\
\hline
24 & HAMID PIRAHESH & 2135         &         &         &$\bullet$&         \\
\hline
25 & BRUCE G. LINDSAY & 2013       &         &         &$\bullet$&         \\
\hline
\end{tabular}
\end{center}

\subsubsection{Vážený outdegree}
\begin{center}
\begin{tabular}{|l|l|c|c|c|c|c|}
\hline
& {\bf Autor} & {\bf outdegree} & {\bf Turing} & {\bf Codd} & {\bf Fellows} & {\bf ISI} \\
\hline
1  & MICHAEL J. CAREY & 3239    &&$\bullet$&$\bullet$&         \\
\hline
2  & GERHARD WEIKUM & 3071      &&         &$\bullet$&         \\
\hline
3  & DAVID J. DEWITT & 2818     &&$\bullet$&$\bullet$&         \\
\hline
4  & PHILIP S. YU & 2614        &&         &$\bullet$&$\bullet$\\
\hline
5  & HECTOR GARCIA-MOLINA & 2512&&$\bullet$&$\bullet$&$\bullet$\\
\hline
6  & MICHAEL STONEBRAKER & 2316 &&$\bullet$&$\bullet$&         \\
\hline
7  & SERGE ABITEBOUL & 2297     &&$\bullet$&$\bullet$&$\bullet$\\
\hline
8  & H. V. JAGADISH & 2263      &&         &$\bullet$&         \\
\hline
9  & RAKESH AGRAWAL & 2240      &&$\bullet$&$\bullet$&         \\
\hline
10 & RAGHU RAMAKRISHNAN & 2059  &&         &$\bullet$&         \\
\hline
11 & CHRISTOS FALOUTSOS & 2042  &&         &$\bullet$&         \\
\hline
12 & WON KIM & 1902             &&         &$\bullet$&         \\
\hline
13 & ABRAHAM SILBERSCHATZ & 1867&&         &$\bullet$&         \\
\hline
14 & MIRON LIVNY & 1806         &&         &         &         \\
\hline
15 & GOETZ GRAEFE & 1789        &&         &         &         \\
\hline
16 & STEFANO CERI & 1775        &&         &         &         \\
\hline
17 & YANNIS E. IOANNIDIS & 1775 &&         &$\bullet$&         \\
\hline
18 & RICHARD HULL & 1692        &&         &$\bullet$&         \\
\hline
19 & HAMID PIRAHESH & 1685      &&         &$\bullet$&         \\
\hline
20 & HANS-JOUMLRG SCHEK & 1661  &&         &$\bullet$&         \\
\hline
21 & STANLEY Y. W. SU & 1651    &&         &         &         \\
\hline
22 & CLEMENT T. YU & 1630       &&         &         &         \\
\hline
23 & JEFFREY F. NAUGHTON & 1587 &&         &$\bullet$&         \\
\hline
24 & RICHARD T. SNODGRASS & 1558&&         &$\bullet$&         \\
\hline
25 & SHAMKANT B. NAVATHE & 1538 &&         &         &         \\
\hline
\end{tabular}
\end{center}


\subsubsection{PageRank}
Hodnoty PageRanku dosahují hodnot mezi $0$ a $1$. Pro účely přehlednosti byly v
této tabulce normalizovány na interval $0$ až $|V|$, tedy počet uzlů sítě.

\begin{center}
\begin{tabular}{|l|l|c|c|c|c|c|}
\hline
& {\bf Autor} & {\bf PageRank} & {\bf Turing} & {\bf Codd} & {\bf Fellows} & {\bf ISI} \\
\hline
1 & E. F. CODD & 179.324            & $\bullet$ &           & $\bullet$ &           \\
\hline
2 & MICHAEL STONEBRAKER & 137.371   &           & $\bullet$ & $\bullet$ &           \\
\hline
3 & JIM GRAY & 115.364              &           & $\bullet$ & $\bullet$ &           \\
\hline
4 & DONALD D. CHAMBERLIN & 114.010  &           &           & $\bullet$ &           \\
\hline
5 & RAYMOND A. LORIE & 107.204      &           &           & $\bullet$ &           \\
\hline
6 & PHILIP A. BERNSTEIN & 99.575    &           & $\bullet$ & $\bullet$ & $\bullet$ \\
\hline
7 & MORTON M. ASTRAHAN & 87.673     &           &           &           &           \\
\hline
8 & KAPALI P. ESWARAN & 87.167      &           &           &           &           \\
\hline
9 & PETER P. CHEN & 84.098          &           &           & $\bullet$ &           \\
\hline
10 & IRVING L. TRAIGER & 79.313     &           &           & $\bullet$ &           \\
\hline
11 & JOHN MILES SMITH & 78.833      &           &           &           &           \\
\hline
12 & JEFFREY D. ULLMAN & 74.323     &           & $\bullet$ & $\bullet$ & $\bullet$ \\
\hline
13 & EUGENE WONG & 68.319           &           &           &           &           \\
\hline
14 & DAVID J. DEWITT & 67.701       &           & $\bullet$ & $\bullet$ &           \\
\hline
15 & MIKE W. BLASGEN & 62.185       &           &           & $\bullet$ &           \\
\hline
16 & GIANFRANCO R. PUTZOLU & 61.585 &           &           &           &           \\
\hline
17 & BRADFORD W. WADE & 60.731      &           &           &           &           \\
\hline
18 & RUDOLF BAYER & 60.706          &           & $\bullet$ &           &           \\
\hline
19 & JAMES W. MEHL & 58.499         &           &           &           &           \\
\hline
20 & PATRICIA P. GRIFFITHS & 58.215 &           &           &           &           \\
\hline
21 & WON KIM & 57.946               &           & $\bullet$ & $\bullet$ &           \\
\hline
22 & W. FRANK KING III & 57.169     &           &           &           &           \\
\hline
23 & NATHAN GOODMAN & 56.791        &           &           &           & $\bullet$ \\
\hline
24 & PAUL R. MCJONES & 55.967       &           &           & $\bullet$ &           \\
\hline
25 & RONALD FAGIN & 54.766          &           & $\bullet$ & $\bullet$ & $\bullet$ \\
\hline
\end{tabular}
\end{center}

\subsubsection{Nevážený closeness}
\begin{center}
\begin{tabular}{|l|l|c|c|c|c|c|}
\hline
& {\bf Autor} & {\bf Closeness} & {\bf Turing} & {\bf Codd} & {\bf Fellows} & {\bf ISI} \\
\hline
1  & MICHAEL STONEBRAKER & 0.593& & $\bullet$ & $\bullet$ &           \\
\hline
2  & JIM GRAY & 0.560&           & $\bullet$ & $\bullet$ &           \\
\hline
3  & DAVID J. DEWITT & 0.556&           & $\bullet$ & $\bullet$ &           \\
\hline
4  & RAYMOND A. LORIE & 0.556&           &           & $\bullet$ &           \\
\hline
5  & JEFFREY D. ULLMAN & 0.546&           & $\bullet$ & $\bullet$ & $\bullet$ \\
\hline
6  & PHILIP A. BERNSTEIN & 0.546&& $\bullet$ & $\bullet$ & $\bullet$ \\
\hline
7  & E. F. CODD & 0.543& $\bullet$ &           & $\bullet$ &           \\
\hline
8  & DONALD D. CHAMBERLIN & 0.539& & & $\bullet$ &           \\
\hline
9  & WON KIM & 0.537&           & $\bullet$ & $\bullet$ &           \\
\hline
10 & UMESHWAR DAYAL & 0.535&         &$\bullet$&$\bullet$&         \\
\hline
11 & MICHAEL J. CAREY & 0.532&&$\bullet$&$\bullet$&         \\
\hline
12 & MORTON M. ASTRAHAN & 0.531&     &     &     &           \\
\hline
13 & DAVID MAIER & 0.529&         &$\bullet$&$\bullet$&$\bullet$\\
\hline
14 & KAPALI P. ESWARAN & 0.529 &    &   &     &    \\
\hline
15 & NATHAN GOODMAN & 0.527 & & $\bullet$ &         &         \\
\hline
16 & EUGENE WONG & 0.526& &  &   & \\
\hline
17 & IRVING L. TRAIGER & 0.525&           &           & $\bullet$ &           \\
\hline
18 & HECTOR GARCIA-MOLINA & 0.523 &&$\bullet$&$\bullet$&$\bullet$\\
\hline
19 & FRANCCEDILOIS BANCILHON & 0.520 &         &         &         &         \\
\hline
20 & BRUCE G. LINDSAY & 0.519&         &         &$\bullet$&         \\
\hline
21 & PETER P. CHEN & 0.518&           &           & $\bullet$ &           \\
\hline
22 & RAKESH AGRAWAL & 0.518&&$\bullet$&$\bullet$&         \\
\hline
23 & RONALD FAGIN & 0.517&           & $\bullet$ & $\bullet$ & $\bullet$ \\
\hline
24 & CATRIEL BEERI & 0.517&         &         &$\bullet$&$\bullet$\\
\hline
25 & THOMAS G. PRICE & 0.514 & & & & \\
\hline
\end{tabular}
\end{center}

\subsubsection{Vážený closeness}
\begin{center}
\begin{tabular}{|l|l|c|c|c|c|c|}
\hline
& {\bf Autor} & {\bf Closeness} & {\bf Turing} & {\bf Codd} & {\bf Fellows} & {\bf ISI} \\
\hline
1  & MICHAEL STONEBRAKER & 2.055& & $\bullet$ & $\bullet$ &           \\
\hline
2  & JIM GRAY & 2.047&           & $\bullet$ & $\bullet$ &           \\
\hline
3  & E. F. CODD & 2.043& $\bullet$ &           & $\bullet$ &           \\
\hline
4  & DAVID J. DEWITT & 2.017&           & $\bullet$ & $\bullet$ &           \\
\hline
5  & JEFFREY D. ULLMAN & 2.014&           & $\bullet$ & $\bullet$ & $\bullet$ \\
\hline
6  & RAYMOND A. LORIE & 2.012&           &           & $\bullet$ &           \\
\hline
7  & PHILIP A. BERNSTEIN & 2.011&& $\bullet$ & $\bullet$ & $\bullet$ \\
\hline
8  & MICHAEL J. CAREY & 1.993&&$\bullet$&$\bullet$&         \\
\hline
9  & DAVID MAIER & 1.979&         &$\bullet$&$\bullet$&$\bullet$\\
\hline
10 & EUGENE WONG & 1.973& &  &   & \\
\hline
11 & DONALD D. CHAMBERLIN & 1.969& & & $\bullet$ &           \\
\hline
12 & LAWRENCE A. ROWE & 1.967 &&$\bullet$&&\\
\hline
13 & NATHAN GOODMAN & 1.966 & & $\bullet$ &         &         \\
\hline
14 & YEHOSHUA SAGIV & 1.966&         &         &         &         \\
\hline
15 & HECTOR GARCIA-MOLINA & 1.960 &&$\bullet$&$\bullet$&$\bullet$\\
\hline
16 & IRVING L. TRAIGER & 1.955&           &           & $\bullet$ &           \\
\hline
17 & CATRIEL BEERI & 1.949&         &         &$\bullet$&$\bullet$\\
\hline
18 & BRUCE G. LINDSAY & 1.944&         &         &$\bullet$&         \\
\hline
19 & MORTON M. ASTRAHAN & 1.943&     &     &     &           \\
\hline
20 & JEFFREY F. NAUGHTON & 1.942&&         &$\bullet$&         \\
\hline
21 & JENNIFER WIDOM & 1.937&         &$\bullet$&$\bullet$&         \\
\hline
22 & RAGHU RAMAKRISHNAN & 1.936 &&         &$\bullet$&         \\
\hline
23 & MIRON LIVNY & 1.936&&         &         &         \\
\hline
24 & RANDY H. KATZ & 1.934 &&$\bullet$&&\\
\hline
25 & RAKESH AGRAWAL & 1.933&&$\bullet$&$\bullet$&         \\
\hline
\end{tabular}
\end{center}

\subsubsection{Navážený betweeness}
\begin{center}
\begin{tabular}{|l|l|c|c|c|c|c|}
\hline
& {\bf Autor} & {\bf Betweeness} & {\bf Turing} & {\bf Codd} & {\bf Fellows} & {\bf ISI} \\
\hline
1 & PHILIP A. BERNSTEIN & 62655703.293 & & $\bullet$ & $\bullet$ & $\bullet$ \\
\hline
2 & MICHAEL STONEBRAKER & 61738362.921 & & $\bullet$ & $\bullet$ &           \\
\hline
3 & DAVID J. DEWITT & 60335509.092     & & $\bullet$ & $\bullet$ &           \\
\hline
4 & JIM GRAY & 58452724.132            & & $\bullet$ & $\bullet$ &           \\
\hline
5 & UMESHWAR DAYAL & 58105048.655      & & $\bullet$ & $\bullet$ &         \\
\hline
6 & RAYMOND A. LORIE & 57606842.228    & &           & $\bullet$ &           \\
\hline
7 & DONALD D. CHAMBERLIN & 57435250.431& &           & $\bullet$ &           \\
\hline
8 & MICHAEL J. CAREY & 56191915.811    & & $\bullet$ & $\bullet$ &         \\
\hline
9 & JEFFREY D. ULLMAN & 56098986.122   & & $\bullet$ & $\bullet$ & $\bullet$ \\
\hline
10 & KAPALI P. ESWARAN & 55953909.624  & &           &           &           \\
\hline
11 & E. F. CODD & 55595773.178         & $\bullet$ & & $\bullet$ &           \\
\hline
12 & WON KIM & 55485910.707            & & $\bullet$ & $\bullet$ &           \\
\hline
13 & MORTON M. ASTRAHAN & 53967137.730 &         &         &         & \\
\hline
14 & DAVID MAIER & 53884993.441        & &$\bullet$&$\bullet$&$\bullet$\\
\hline
15 & FRANCCEDILOIS BANCILHON & 52436978.786 &  &  & &         \\
\hline
16 & NATHAN GOODMAN & 51776071.388& & $\bullet$ &         &         \\
\hline
17 & EUGENE WONG & 50457002.386&         &         &         & \\
\hline
18 & IRVING L. TRAIGER & 50067735.663&         &         &$\bullet$& \\
\hline
19 & HECTOR GARCIA-MOLINA & 49279794.248& & $\bullet$ & $\bullet$ & $\bullet$ \\
\hline
20 & CATRIEL BEERI & 49031169.516&         &         &$\bullet$&$\bullet$\\
\hline
21 & RONALD FAGIN & 48476621.189& & $\bullet$ & $\bullet$ & $\bullet$ \\
\hline
22 & BRUCE G. LINDSAY & 47956637.448&         &         &$\bullet$&         \\
\hline
23 & SERGE ABITEBOUL & 47196023.670& & $\bullet$ & $\bullet$ & $\bullet$ \\
\hline
24 & RAKESH AGRAWAL & 46621125.945&&$\bullet$&$\bullet$&         \\
\hline
25 & PATRICIA G. SELINGER & 45312957.343 & & $\bullet$ & $\bullet$ & \\ 
\hline
\end{tabular}
\end{center}


\subsubsection{Vážený betweeness}
\begin{center}
\begin{tabular}{|l|l|c|c|c|c|c|}
\hline
& {\bf Autor} & {\bf Betweeness} & {\bf Turing} & {\bf Codd} & {\bf Fellows} & {\bf ISI} \\
\hline
1  & MICHAEL STONEBRAKER & 51920270.604 & & $\bullet$ & $\bullet$ &   \\
\hline
2  & DAVID J. DEWITT & 47407633.796& & $\bullet$ & $\bullet$ &           \\
\hline
3  & JIM GRAY & 46202513.744& & $\bullet$ & $\bullet$ &           \\
\hline
4  & JEFFREY D. ULLMAN & 43255880.998  & & $\bullet$ & $\bullet$ & $\bullet$ \\
\hline
5  & MICHAEL J. CAREY & 40700171.932 & & $\bullet$ & $\bullet$ &         \\
\hline
6  & RAYMOND A. LORIE & 37350006.114 & &           & $\bullet$ &           \\
\hline
7  & PHILIP A. BERNSTEIN & 36361815.206 & & $\bullet$ & $\bullet$ & $\bullet$ \\
\hline
8  & LAWRENCE A. ROWE & 35035718.542 &&$\bullet$&&\\
\hline
9  & EUGENE WONG & 34499259.565&         &         &         & \\
\hline
10 & MIRON LIVNY & 34488016.860&&         &         &         \\
\hline
11 & YEHOSHUA SAGIV & 32954711.980& &         &         &         \\
\hline
12 & DONALD D. CHAMBERLIN & 32704585.298& &           & $\bullet$ &           \\
\hline
13 & C. MOHAN & 32417070.578& & $\bullet$ & $\bullet$ &         \\
\hline
14 & DAVID MAIER & 32132542.674& &$\bullet$&$\bullet$&$\bullet$\\
\hline
15 & NATHAN GOODMAN & 31719323.894& & $\bullet$ &         &         \\
\hline
16 & HECTOR GARCIA-MOLINA & 31080457.354& & $\bullet$ & $\bullet$ & $\bullet$ \\
\hline
17 & RANDY H. KATZ & 30977276.737 &&$\bullet$&&\\
\hline
18 & JENNIFER WIDOM & 30793150.109&         &$\bullet$&$\bullet$& \\
\hline
19 & RAKESH AGRAWAL & 30763284.150&&$\bullet$&$\bullet$&         \\
\hline
20 & E. F. CODD & 29634524.870& $\bullet$ & & $\bullet$ &           \\
\hline
21 & JEFFREY F. NAUGHTON & 29267110.900&&         &$\bullet$&         \\
\hline
22 & HAMID PIRAHESH & 29233790.884&&         &$\bullet$&         \\
\hline
23 & CATRIEL BEERI & 28857797.364&         &         &$\bullet$&$\bullet$\\
\hline
24 & BRUCE G. LINDSAY & 28566231.162&         &         &$\bullet$&         \\
\hline
25 & RAGHU RAMAKRISHNAN & 28057226.481 &&         &$\bullet$&         \\
\hline
\end{tabular}
\end{center}



\subsection{CiteSeer}

\subsubsection{PageRank}
\begin{center}
\begin{tabular}{|l|l|c|c|c|c|c|}
\hline
& {\bf Autor} & {\bf PageRank} & {\bf Turing} & {\bf Codd} & {\bf Fellows} & {\bf ISI} \\
\hline
1  & JOHN K. OUSTERHOUT & 413.275 &         &         &$\bullet$&         \\
2  & MARTIN E. HELLMAN & 336.552  &         &         &         &         \\
3  & WHITFIELD DIFFIE & 289.814   &         &         &         &         \\
4  & SENIOR MEMBER & 280.899      &         &         &         &         \\
5  & JACK J. DONGARRA & 279.157   &         &         &$\bullet$&$\bullet$\\
6  & VAN JACOBSON & 259.762       &         &         &         &         \\
7  & SCOTT SHENKER & 225.241      &         &         &$\bullet$&         \\
8  & S. KENT & 224.260            &         &         &         &         \\
9  & RANDAL E. BRYANT & 197.141   &         &         &$\bullet$&$\bullet$\\
10 & SALLY FLOYD & 196.267        &         &         &$\bullet$&$\bullet$\\
11 & LIXIA ZHANG & 194.574        &         &         &$\bullet$&         \\
12 & STUDENT MEMBER & 182.926     &         &         &         &         \\
13 & S. KIRKPATRICK & 181.416     &         &         &$\bullet$&$\bullet$\\
14 & C. D. GELATT & 181.416       &         &         &         &         \\
15 & M. P. VECCHI & 181.416       &         &         &         &         \\
16 & TAKEO KANADE & 178.883       &         &         &$\bullet$&         \\
17 & RANDOLPH BENTSON & 178.869   &         &         &         &         \\
18 & GEORGE W. FURNAS & 177.145   &         &         &$\bullet$&         \\
19 & RAKESH AGRAWAL & 175.358     &         &$\bullet$&$\bullet$&         \\
20 & DEBORAH ESTRIN & 173.030     &         &         &$\bullet$&         \\
21 & STEPHEN C. JOHNSON & 168.657 &         &         &         &         \\
22 & EDWARD H. ADELSON & 162.659  &         &         &         &$\bullet$\\
23 & KEN THOMPSON & 159.405       &$\bullet$&         &         &         \\
24 & ADI SHAMIR & 155.899         &$\bullet$&         &         &         \\
25 & MICHAEL J. KARELS & 153.567  &         &         &         &         \\
\hline
\end{tabular}
\end{center}

\subsubsection{Nevážený closeness}
\begin{center}
\begin{tabular}{|l|l|c|c|c|c|c|}
\hline
& {\bf Autor} & {\bf Closeness} & {\bf Turing} & {\bf Codd} & {\bf Fellows} & {\bf ISI} \\
\hline
1  & SENIOR MEMBER & 0.393          & &         &         &          \\
2  & JOHN K. OUSTERHOUT & 0.392     & &         &$\bullet$&          \\
3  & SCOTT SHENKER & 0.384          & &         &$\bullet$&          \\
4  & M. FRANS KAASHOEK & 0.383      & &         &$\bullet$&          \\
5  & STUDENT MEMBER & 0.380         & &         &         &          \\
6  & RAKESH AGRAWAL & 0.380         & &$\bullet$&$\bullet$&          \\
7  & HARI BALAKRISHNAN & 0.377      & &         &$\bullet$&          \\
8  & DEBORAH ESTRIN & 0.377         & &         &$\bullet$&          \\
9  & HECTOR GARCIA-MOLINA & 0.376   & &$\bullet$&$\bullet$&$\bullet$ \\
10 & FACHBEREICH INFORMATIK & 0.375 & &         &         &          \\
11 & VAN JACOBSON & 0.375           & &         &         &          \\
12 & RAJEEV MOTWANI & 0.374         & &         &$\bullet$&          \\
13 & SALLY FLOYD & 0.373            & &         &$\bullet$&$\bullet$ \\
14 & DAVID CULLER & 0.370           & &         &$\bullet$&          \\
15 & LIXIA ZHANG & 0.370            & &         &$\bullet$&          \\
16 & CHRISTOS FALOUTSOS & 0.370     & &         &$\bullet$&          \\
17 & IAN FOSTER & 0.370             & &         &$\bullet$&$\bullet$ \\
18 & STEVEN MCCANNE & 0.370         & &         &         &          \\
19 & PRABHAKAR RAGHAVAN & 0.369     & &         &$\bullet$&$\bullet$ \\
20 & JENNIFER WIDOM & 0.369         & &$\bullet$&$\bullet$&          \\
21 & ROBERT E. SCHAPIRE & 0.368     & &         &         &$\bullet$ \\
22 & ROBERT MORRIS & 0.368          & &         &         &          \\
23 & M. SATYANARAYANAN & 0.368      & &         &$\bullet$&          \\
24 & PETER B. DANZIG & 0.367        & &         &         &          \\
25 & VERN PAXSON & 0.367            & &         &$\bullet$&$\bullet$ \\
\hline
\end{tabular}
\end{center}

\subsubsection{Nevážený betweeness}
\begin{center}
\begin{tabular}{|l|l|c|c|c|c|c|}
\hline
& {\bf Autor} & {\bf Betweeness} & {\bf Turing} & {\bf Codd} & {\bf Fellows} & {\bf ISI} \\
\hline
1  & M. FRANS KAASHOEK & 10112159061.330 & &         &$\bullet$&         \\
2  & SCOTT SHENKER & 9785892051.377      & &         &$\bullet$&         \\
3  & SENIOR MEMBER & 8845140725.909      & &         &         &         \\
4  & VAN JACOBSON & 8813158813.753       & &         &         &         \\
5  & SALLY FLOYD & 8690842977.231        & &         &$\bullet$&$\bullet$\\
6  & LARRY L. PETERSON & 8630281410.114  & &         &$\bullet$&         \\
7  & HARI BALAKRISHNAN & 8544868651.846  & &         &$\bullet$&         \\
8  & JENNIFER WIDOM & 8512314665.858     & &$\bullet$&$\bullet$&         \\
9  & DEBORAH ESTRIN & 8414557973.610     & &         &$\bullet$&         \\
10 & MONICA S. LAM & 8394649393.784      & &         &$\bullet$&         \\
11 & LIXIA ZHANG & 8350916122.773        & &         &$\bullet$&         \\
12 & STEVEN MCCANNE & 8263572085.250     & &         &         &         \\
13 & M. SATYANARAYANAN & 8087193503.971  & &         &$\bullet$&         \\
14 & THOMAS E. ANDERSON & 8078380316.694 & &         &$\bullet$&         \\
15 & DON TOWSLEY & 8053219720.880        & &         &$\bullet$&$\bullet$\\
16 & JOHN K. OUSTERHOUT & 8039121635.247 & &         &$\bullet$&         \\
17 & PETER B. DANZIG & 7986359949.439    & &         &         &         \\
18 & SERGE ABITEBOUL & 7924035872.338    & &$\bullet$&$\bullet$&$\bullet$\\
19 & CHRISTOS FALOUTSOS & 7915737482.408 & &         &$\bullet$&         \\
20 & STUDENT MEMBER & 7854847262.677     & &         &         &         \\
21 & KEN KENNEDY & 7846470147.904        & &         &$\bullet$&$\bullet$\\
22 & Y H. KATZ & 7746528995.356          & &         &$\bullet$&         \\
23 & DAVID B. JOHNSON & 7657508456.644   & &         &         &         \\
24 & RAKESH AGRAWAL & 7615283090.821     & &$\bullet$&$\bullet$&         \\
25 & HUI ZHANG & 7588067468.401          & &         &$\bullet$&         \\
\hline
\end{tabular}
\end{center}


\begin{center}
\begin{tabular}{|l|l|c|c|c|c|c|}
\hline
& {\bf Autor} & {\bf H-index} & {\bf Turing} & {\bf Codd} & {\bf Fellows} & {\bf ISI} \\
\hline
1  & SCOTT SHENKER & 37        & &         &$\bullet$&         \\
2  & DEBORAH ESTRIN & 34       & &         &$\bullet$&         \\
3  & KEN KENNEDY & 33          & &         &$\bullet$&$\bullet$\\
4  & DOUGLAS C. SCHMIDT & 33   & &         &         &         \\
5  & DON TOWSLEY & 32          & &         &$\bullet$&$\bullet$\\
6  & HECTOR GARCIA-MOLINA & 31 & &$\bullet$&$\bullet$&$\bullet$\\
7  & THOMAS A. HENZINGER & 30  & &         &$\bullet$&$\bullet$\\
8  & RAKESH AGRAWAL & 29       & &$\bullet$&$\bullet$&         \\
9  & M. FRANS KAASHOEK & 29    & &         &$\bullet$&         \\
10 & JENNIFER WIDOM & 29       & &$\bullet$&$\bullet$&         \\
11 & WILLY ZWAENEPOEL & 28     & &         &$\bullet$&         \\
12 & HUI ZHANG & 27            & &         &$\bullet$&         \\
13 & IAN FOSTER & 27           & &         &$\bullet$&$\bullet$\\
14 & MONI NAOR & 27            & &         &         &         \\
15 & SALLY FLOYD & 26          & &         &$\bullet$&$\bullet$\\
16 & LUCA CARDELLI & 26        & &         &$\bullet$&         \\
17 & SERGE ABITEBOUL & 26      & &$\bullet$&$\bullet$&$\bullet$\\
18 & SENIOR MEMBER & 26        & &         &         &         \\
19 & BART SELMAN & 26          & &         &$\bullet$&         \\
20 & SEBASTIAN THRUN & 25      & &         &         &         \\
21 & OREN ETZIONI & 24         & &         &         &         \\
22 & DAVID J. DEWITT & 24      & &$\bullet$&$\bullet$&         \\
23 & DAPHNE KOLLER & 24        & &         &         &         \\
24 & RAJEEV ALUR & 24          & &         &$\bullet$&$\bullet$\\
25 & HARI BALAKRISHNAN & 24    & &         &$\bullet$&         \\
\hline
\end{tabular}
\end{center}



\chapter{Diskuse}
\section{Podobnost výsledků jednotlivých metod}
\section{Shoda výsledků s oceněními}
\section{Vliv vah na přesnost výsledků}
\section{Vstupní a výstupní hrany}


\chapter{Závěr}

\bibliographystyle{alpha}
\bibliography{references}
\end{document}
