\documentclass[12pt,titlepage]{report}
\usepackage[utf8]{inputenc}
\usepackage[czech]{babel}
\usepackage{amsmath}
\usepackage{graphicx}
\usepackage{microtype}
\usepackage[Sonny]{fncychap}

\usepackage[pdfborder=0 0 0]{hyperref}

\begin{document}
\begin{titlepage}
\begin{center}
	\Large{Západočeská univerzita v Plzni} \\
	\Large{Fakulta aplikovaných věd} \\
	\Large{Katedra infromatiky a výpočetní techniky} \\
\mbox{} \\[1.6cm]
	\LARGE{{\bf Bakalářská práce}} \\
\mbox{} \\
	\Huge{{\bf Měření významnosti autorů v citační síti}} \\
\end{center}
\vfill
\begin{minipage}{.5\textwidth}
Plzeň, 2013
\end{minipage}
\begin{minipage}{.5\textwidth}
\hfill Tomáš Maršálek
\end{minipage}
\thispagestyle{empty}
\end{titlepage}

\section*{Abstract}

\tableofcontents

\chapter{Úvod}

\chapter{Sociální a citační sítě}
\section{Sociální sítě}

\section{Analýza sociálních sítí}

\section{Citační sítě}
Citační sítě jsou speciálním případem sociálních sítí, kde jako uzly uvažujeme
knihy, články nebo jiné publikace; nebo uzly mohou být samotní autoři těchto
publikací. Spojení mezi uzly je určeno citacemi na jednotlivé publikace. 

\subsection{Síť publikací}
Uvažujeme-li první případ, kde uzly reprezentují publikace a hrany přímo citace
mezi těmito publikacemi, jedná se o síť publikací. Tedy pokud publikace $A$
odkazuje na publikaci $B$, pak budou existovat stejnojmenné uzly $A$ a $B$ a
hrana mezi těmito uzly může mít dvě různé orientace podle svého uplatnění. Směr
od citující publikace k citované (v našem příkladě od $A$ do $B$) bude mít
hrana, kterou označíme jako výstupní pro uzel $A$ a vstupní pro uzel $B$.
Výstupní hrana laicky řečeno označuje vztah "cituji", kdežto vstupní hrana
znamená "jsem citován".

\subsection{Síť autorů}
Druhým případem citační sítě je síť autorů. Zde je uzel reprezentací autora a
hrany spojují autory mezi sebou. Ve většině případech máme k dispozici data ve
formátu, který přímo odpovídá síti publikací, tzn. pro jednu publikaci známe
seznam jejích autorů a odkazů na další publikace. Síť autorů lze získat
transformací sítě publikací tak, že každou hranu z původní sítě publikací
přiřadíme každému z autorů této publikace a duplikujeme ji pro každého z autorů
citované publikace. Celkově vznikne $nm$ nových hran, pokud odkazovaná
publikace obsahuje $n$ autorů a odkazující $m$ autorů. Stejně jako v síti
publikací, i zde uvažujeme dvě opačné orientace hrany se stejnou interpretací,
tedy "cituji" a "jsem citován". 

V síti autorů má pro naše účely smysl uvažovat
ohodnocení hran. Existuje více způsobů, jak přiřadit ohodnocení (váhy)
jednotlivým hranám, ale nejjednodušším způsobem, který je použitý i v
implementaci knihovny, je prosté přiřazení počtu publikací, jejichž autorem
nebo spoluautorem je daný autor $A$, které odkazují na publikace, jejichž
autorem je autor $B$. Srozumitelnější popis poskytne obrázek:
%% TODO figure váha hran.

Druhým způsobem ohodnocení hran, který rovněž využívá implementovaná knihovna
pro některé metody, je převrácená hodnota prvního způsobu ohodnocení. Důvodem
je přímá souvislost mezi vahou hrany a vzdáleností mezi uzly. V prvním případě,
kdy silnější pouto mezi autory vyjadřuje vyšší ohodnocení hrany, v druhém
případě je naopak nižší váha vyjádřením silnějšího vztahu, jelikož jsou si uzly
blíže. Tento způsob je používán pro algoritmy, které pracují na myšlence
nejkratších cest mezi uzly. 

\section{Významnost autorů}
%% TODO tady se hodně zmínit o práci předchozích lidí, ne?

\section{Míry centrality}
\subsection{Degree}
\subsection{Eigenvector}
\subsection{Closeness}
\subsection{Betweeness}

\section{Ostatní míry významnosti autorů}
\subsection{H-index}

\section{Metody porovnaní}
\subsection{Spearmanův koeficient pořadové korelace}
\subsection{Pearsonův korelační koeficient}

\section{Ocenění významných autorů}
\subsection{ACM A.M. Turing Award}
ACM A.M. Turing Award je ocenění ročně udělované skupinou ACM (Association for
Computing Machinery) jedincům vybraným pro kontribuce technického ducha do
výpočetního světa.
\cite{turingaward}.

Turingova cena je brána jako nejvyšší vyznamenání v informatice a je lidově
nazývána Nobelovou cenou pro informatiku \cite[p.~317]{dasgupta}.

\subsection{ACM SIGMOD Edgar F. Codd Innovations Award}
ACM SIGMOD Edgar F. Codd Innovations Award je ohodnocení životního díla
skupinou ACM SIGMOD (Special Interest Group on Management of Data)  za
inovativní a vysoce ceněné kontribuce k rozvoji, porozumění a použití
databázových systémů a databází \cite{sigmodinnovations}.

\subsection{ACM Fellows}
\uv{The ACM Fellows Program} byl založen v roce 1993, aby našel a ocenil
vynikající členy ACM za jejich dílo v informatice a informační vědě a pro
jejich významné kontribuce pro účel ACM. Členové ACM Fellows slouží jako
význační kolegové, ke kterým ACM a jejich členové vzhlížejí jako k autoritám v
době rozvoje informačních technologií \cite{acmfellows}.

\subsection{ISI Highly Cited highlighted}
ISI Highly Cited je databáze často citovaných autorů v článcích posledního
desetiletí, které byly vydány institutem ISI (Institute for Scientific
Information). Ten v dnešní době spadá pod agenturu Thomson Reuters, na jejíchž webových stránkách nalezneme seznam autorů ISI Highly Cited highlighted z let 2000 až 2008 napříč 21 vědeckými obory \cite{highlycited}.

\chapter{Implementace}
\section{Načtení vstupních dat}

\section{Vytvoření citačních sítí}

\section{Analýza struktury sítě}

\section{Knihovna pro SNA}

\subsection{Reprezentace citační sítě}
\subsection{Degree}
\subsection{Eigenvector}
\subsection{Closeness}
\subsection{Betweeness}
\subsection{H-index}

\chapter{Výsledky}
\section{Citační databáze}
\subsection{DBLP}
DBLP \cite{DBLP} je webová bibliografická databáze v oboru informatiky,
která k listopadu 2012 obsahovala více než 2,1 milionu publikací. Pro tuto
práci používáme verzi z roku 2004.

\subsection{CiteSeer}
CiteSeer (nyní CiteSeer$^X$) \cite{citeseer} je považován za první
automatizovaný systém shromažďování publikací a autonomní indexace citací v
nich obsažených. Publikace jsou zejména z oboru informatiky a informační vědy.
V dnešní době obsahuje přes dva miliony dokumentů s téměř dvěma miliony autorů
a čtyřiceti miliony citací. Zde používáme verzi z roku 2005.

\section{Struktura sítě}
\subsection{DBLP}
\subsection{CiteSeer}

\section{Žebříčky významných autorů}
\subsection{DBLP}
\subsubsection{H-index}
\begin{center}
\begin{tabular}{|l|l|c|c|c|c|c|}
\hline
& {\bf Autor} & {\bf H-index} & {\bf Turing} & {\bf Codd} & {\bf Fellows} & {\bf ISI} \\
\hline
1 & MICHAEL STONEBRAKER & 28.000  & & $\bullet$ & $\bullet$ &         \\
\hline
2 & DAVID J. DEWITT & 24.000      & & $\bullet$ & $\bullet$ &         \\
\hline
3 & JEFFREY D. ULLMAN & 24.000    & & $\bullet$ & $\bullet$ & $\bullet$ \\
\hline
4 & PHILIP A. BERNSTEIN & 22.000  & & $\bullet$ & $\bullet$ & $\bullet$ \\
\hline
5 & RAKESH AGRAWAL & 21.000       & & $\bullet$ & $\bullet$ &         \\
\hline
6 & WON KIM & 21.000              & & $\bullet$ & $\bullet$ &         \\
\hline
7 & YEHOSHUA SAGIV & 20.000       & &         &         &         \\
\hline
8 & CATRIEL BEERI & 20.000        & &         & $\bullet$ & $\bullet$ \\
\hline
9 & MICHAEL J. CAREY & 20.000     & & $\bullet$ & $\bullet$ &         \\
\hline
10 & SERGE ABITEBOUL & 19.000      & & $\bullet$ & $\bullet$ & $\bullet$ \\
\hline
11 & HECTOR GARCIA-MOLINA & 19.000 & & $\bullet$ & $\bullet$ & $\bullet$ \\
\hline
12 & UMESHWAR DAYAL & 19.000       & & $\bullet$ & $\bullet$ &         \\
\hline
13 & CHRISTOS FALOUTSOS & 19.000   & &         & $\bullet$ & $\bullet$ \\
\hline
14 & NATHAN GOODMAN & 18.000       & & $\bullet$ &         &         \\
\hline
15 & JIM GRAY & 18.000             & &         & $\bullet$ &         \\
\hline
16 & JEFFREY F. NAUGHTON & 18.000  & &         & $\bullet$ &         \\
\hline
17 & RAGHU RAMAKRISHNAN & 18.000   & &         &         &         \\
\hline
18 & RONALD FAGIN & 18.000         & & $\bullet$ & $\bullet$ & $\bullet$ \\
\hline
19 & JENNIFER WIDOM & 18.000       & & $\bullet$ & $\bullet$ &         \\
\hline
20 & DAVID MAIER & 17.000          & & $\bullet$ & $\bullet$ & $\bullet$ \\
\hline
21 & BRUCE G. LINDSAY & 17.000     & &         & $\bullet$ &         \\
\hline
22 & SHAMKANT B. NAVATHE & 16.000  & &         &         &         \\
\hline
23 & C. MOHAN & 16.000             & & $\bullet$ & $\bullet$ &         \\
\hline
24 & HAMID PIRAHESH & 16.000       & &         & $\bullet$ &         \\
\hline
25 & H. V. JAGADISH & 16.000       & &         & $\bullet$ &         \\
\hline
\end{tabular}
\end{center}

\subsubsection{Degree}
\subsubsection{PageRank}
Hodnoty PageRanku dosahují hodnot mezi $0$ a $1$. Pro účely přehlednosti byly v
této tabulce normalizovány na interval $0$ až $|V|$, tedy počet uzlů sítě.

\begin{center}
\begin{tabular}{|l|l|c|c|c|c|c|}
\hline
& {\bf Autor} & {\bf PageRank} & {\bf Turing} & {\bf Codd} & {\bf Fellows} & {\bf ISI} \\
\hline
1 & E. F. CODD & 179.324            & $\bullet$ &           & $\bullet$ &           \\
\hline
2 & MICHAEL STONEBRAKER & 137.371   &           & $\bullet$ & $\bullet$ &           \\
\hline
3 & JIM GRAY & 115.364              &           & $\bullet$ & $\bullet$ &           \\
\hline
4 & DONALD D. CHAMBERLIN & 114.010  &           &           & $\bullet$ &           \\
\hline
5 & RAYMOND A. LORIE & 107.204      &           &           & $\bullet$ &           \\
\hline
6 & PHILIP A. BERNSTEIN & 99.575    &           & $\bullet$ & $\bullet$ & $\bullet$ \\
\hline
7 & MORTON M. ASTRAHAN & 87.673     &           &           &           &           \\
\hline
8 & KAPALI P. ESWARAN & 87.167      &           &           &           &           \\
\hline
9 & PETER P. CHEN & 84.098          &           &           & $\bullet$ &           \\
\hline
10 & IRVING L. TRAIGER & 79.313     &           &           & $\bullet$ &           \\
\hline
11 & JOHN MILES SMITH & 78.833      &           &           &           &           \\
\hline
12 & JEFFREY D. ULLMAN & 74.323     &           & $\bullet$ & $\bullet$ & $\bullet$ \\
\hline
13 & EUGENE WONG & 68.319           &           &           &           &           \\
\hline
14 & DAVID J. DEWITT & 67.701       &           & $\bullet$ & $\bullet$ &           \\
\hline
15 & MIKE W. BLASGEN & 62.185       &           &           & $\bullet$ &           \\
\hline
16 & GIANFRANCO R. PUTZOLU & 61.585 &           &           &           &           \\
\hline
17 & BRADFORD W. WADE & 60.731      &           &           &           &           \\
\hline
18 & RUDOLF BAYER & 60.706          &           & $\bullet$ &           &           \\
\hline
19 & JAMES W. MEHL & 58.499         &           &           &           &           \\
\hline
20 & PATRICIA P. GRIFFITHS & 58.215 &           &           &           &           \\
\hline
21 & WON KIM & 57.946               &           & $\bullet$ & $\bullet$ &           \\
\hline
22 & W. FRANK KING III & 57.169     &           &           &           &           \\
\hline
23 & NATHAN GOODMAN & 56.791        &           &           &           & $\bullet$ \\
\hline
24 & PAUL R. MCJONES & 55.967       &           &           & $\bullet$ &           \\
\hline
25 & RONALD FAGIN & 54.766          &           & $\bullet$ & $\bullet$ & $\bullet$ \\
\hline
\end{tabular}
\end{center}

\subsubsection{Closeness}
\subsubsection{Betweeness}
\subsubsection{H-index}

\section{Porovnání implementovaných metod}

\chapter{Diskuse}
\section{Podobnost výsledků jednotlivých metod}

\section{Shoda výsledků s oceněními}

\section{Vliv vah na přesnost výsledků}

\section{Vstupní a výstupní hrany}


\chapter{Závěr}

\bibliographystyle{alpha}
\bibliography{references}
\end{document}
