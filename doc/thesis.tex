\documentclass[12pt,titlepage]{report}
\usepackage[utf8]{inputenc}
\usepackage[czech]{babel}
\usepackage{amsmath}
\usepackage{graphicx}
\usepackage{microtype}
\usepackage[Sonny]{fncychap}

\usepackage[pdfborder=0 0 0]{hyperref}

\begin{document}
\begin{titlepage}
\begin{center}
	\Large{Západočeská univerzita v Plzni} \\
	\Large{Fakulta aplikovaných věd} \\
	\Large{Katedra infromatiky a výpočetní techniky} \\
\mbox{} \\[1.6cm]
	\LARGE{{\bf Bakalářská práce}} \\
\mbox{} \\
	\Huge{{\bf Měření významnosti autorů v citační síti}} \\
\end{center}
\vfill
\begin{minipage}{.5\textwidth}
Plzeň, 2013
\end{minipage}
\begin{minipage}{.5\textwidth}
\hfill Tomáš Maršálek
\end{minipage}
\thispagestyle{empty}
\end{titlepage}

\section*{Abstrakt}
Prvky sociální sítě, které nemají žádné apriorní ohodnocení významnosti, jsou
různě významné pouze na základě vztahů s okolními prvky. V této práci byly
prozkoumány a implementovány známé metody centrality a bibliografické metody
měřící významnost prvků v sociální nebo citační síti. Výsledky aplikování metod
na volně dostupné citační databáze ukázaly vysokou podobnost jednotlivých metod
a rovněž shodu nejvýznamnějších autorů dle těchto metod se známými oceněními v
oblasti informatiky a informační vědy (Turing Award, Codd, ACM Fellows, ISI
Highly Cited). Bylo zjištěno, že některé implementované metody jsou i přes
použití nejrychlejších algoritmů výpočetně příliš náročné vzhledem k velikosti
citačních sítí, vzniklých z těchto citačních databází. Dále bylo empiricky
potvrzeno, že implementované metody měří významnost, byť může mít více
interpretací.

\tableofcontents

\chapter{Úvod}

\section{Sociální sítě}

\subsection{Analýza sociálních sítí}

\section{Citační sítě}
Citační sítě jsou podobné sociálním sítím, pouze místo uzlů, které představují
osoby, v citační síti se jedná o publikace nebo autory těchto publikací.  Pokud
je uzlem publikace, pak hrany této sítě symbolizují citaci publikace jinou
publikací. V druhém případě uvažujeme síť, kde uzly reprezentují autory knih,
vědeckých článků, vědecké literatury a dalších publikací. Prvnímu typu říkáme
síť publikací, druhému síť autorů.

\subsection{Síť publikací}
Uvažujeme-li první případ, kde uzly reprezentují publikace a hrany přímo citace
mezi těmito publikacemi, jedná se o síť publikací. Tedy pokud publikace $A$
odkazuje na publikaci $B$, pak budou existovat stejnojmenné uzly $A$ a $B$ a
hrana mezi těmito uzly může mít dvě různé orientace podle svého uplatnění. Směr
od citující publikace k citované (v našem příkladě od $A$ do $B$) bude mít
hrana, kterou označíme jako výstupní pro uzel $A$ a vstupní pro uzel $B$.
Výstupní hrana laicky řečeno označuje vztah "cituji", kdežto vstupní hrana
znamená "jsem citován".

\subsection{Síť autorů}
Druhým případem citační sítě je síť autorů. Zde je uzel reprezentací autora a
hrany spojují autory mezi sebou. Ve většině případech máme k dispozici data ve
formátu, který přímo odpovídá síti publikací, tzn. pro jednu publikaci známe
seznam jejích autorů a odkazů na další publikace. Síť autorů lze získat
transformací sítě publikací tak, že každou hranu z původní sítě publikací
přiřadíme každému z autorů této publikace a duplikujeme ji pro každého z autorů
citované publikace. Celkově vznikne $nm$ nových hran, pokud odkazovaná
publikace obsahuje $n$ autorů a odkazující $m$ autorů. Stejně jako v síti
publikací, i zde uvažujeme dvě opačné orientace hrany se stejnou interpretací,
tedy "cituji" a "jsem citován". 

V síti autorů má pro naše účely smysl uvažovat
ohodnocení hran. Existuje více způsobů, jak přiřadit ohodnocení (váhy)
jednotlivým hranám, ale nejjednodušším způsobem, který je použitý i v
implementaci knihovny, je prosté přiřazení počtu publikací, jejichž autorem
nebo spoluautorem je daný autor $A$, které odkazují na publikace, jejichž
autorem je autor $B$. Srozumitelnější popis poskytne obrázek:
%% TODO figure váha hran.

Druhým způsobem ohodnocení hran, který rovněž využívá implementovaná knihovna
pro některé metody, je převrácená hodnota prvního způsobu ohodnocení. Důvodem
je přímá souvislost mezi vahou hrany a vzdáleností mezi uzly. V prvním případě,
kdy silnější pouto mezi autory vyjadřuje vyšší ohodnocení hrany, v druhém
případě je naopak nižší váha vyjádřením silnějšího vztahu, jelikož jsou si uzly
blíže. Tento způsob je používán pro algoritmy, které pracují na myšlence
nejkratších cest mezi uzly. 

\subsection{Vážené citační sítě}
V definici grafu nebo sítě $G = (V, E)$ je množina hran $E$ soubor dvojic,
které označují koncové uzly hrany, neboli jejich spojení. Samotné spojení je
jediná informace, kterou množina hran nese. Chceme-li zaznamenat nějakou další
informaci, která je spojená se spojením dvou uzlů, namísto hrany jako dvojice
koncových uzlů nadefinujeme hranu jako n-tici, kde první dvě hodnoty jsou
koncové uzly a zbylé hodnoty nesou libovolnou informaci. Ve většině případů si
vystačíme s jednou dodatečnou informací a nazýváme ji váha hrany.

Při zavedení vah máme například možnost používat síť jako multigraf, tedy graf,
u kterého je povoleno více spojení mezi dvěma stejnými uzly. Počet stejných
hran pak pouze zaznamenáme celočíselnou hodnotou ve váze hrany.

Například síť world wide web tvořená webovými stránkami je příkladem
multigrafu, protože je povoleno z jedné stránky odkazovat na jinou na více
místech. Při analýze takových sítí využijeme právě vah hran a počet
hypertextových odkazů mezi dvěma stránkami zaznamenáme vyšším ohodnocením
hrany. V tomhle případě znamená vyšší váha silnější pouto mezi uzly.

Jiným případem může být například síť kde sledujeme města a dopravní spojení
mezi nimi. V tomhle případě nás může zajímat vzdálenost nebo časová náročnost
na dopravu mezi dvěma městy, které budou znamenat silnější pouto pokud budou
mít naopak menší váhu. Hledáme totiž nejkratší a nejrychlejší spojení.

Pro citační sítě můžeme uvažovat ohodnocení hran obojího typu. Například mezi
dvěma autory může být silnější vztah, pokud se citují ve více publikacích.
Pokud citační síť analyzujeme metodami, které jsou založené na myšlence hledání
nejkratších cest i v této síti, která nemá v podstatě žádný pojem vzdálenosti,
použijeme druhý typ ohodnocení - menší váha, silnější pouto.


\subsection{Orientované a neorientované sítě}
Obecně můžeme uvažovat grafy s hranami s orientací či bez orientace. V obou
případech se stále jedná o množinu $(V, E)$, pouze pro orientovaný graf je
množina hran množinou uspořádaných dvojic oproti množině neuspořádaných dvojic
u neorientovaného grafu.

Hrany se uzlu v případě orientovaného grafu liší z pohledu jednoho uzlu. Pokud
hrana vychází z tohoto uzlu, nazveme ji výstupní hrana, v opačném případě se
bude jednat o vstupní hranu.

V případě sociálních sítí nejčastěji uvažujeme sítě bez orientace, protože
nejčastěji modelovaný vztah přítel-přítel je ekvivalentní z pohledu obou
koncových uzlů. Pro citační síť jsou na místě orientované hrany, protože vztahy
autor odkazujícího na jiného autora nebo publikace citující jinou publikaci
mají očividně jinou interpretaci z pohledu koncových uzlů. Buďto se jedná o
citovaného nebo citujícího autora či publikaci.

\subsection{Komponenty grafu}
\subsubsection{Silně spojité komponenty}
\subsection{Klika v grafu}

\section{Analýza citačních sítí}
\section{Významnost autorů}
%% TODO tady se hodně zmínit o práci předchozích lidí, ne?

\section{Míry centrality}
\subsection{Degree}
Pro orientovaný graf můžeme uvažovat vstupní (indegree) a výstupní stupeň
(outdegree) vrcholu nebo obecný stupeň (degree), tedy součet těchto dvou. Pokud
uvažujeme pouze vstupní stupeň, vypočtená hodnota určuje významnost uzlu,
kdežto výstupní stupeň ukazuje společenskost či otevřenost uzlu. Obecný stupeň
znamená celkový počet spojení s okolními prvky sítě.

Degree centrality je výpočetně velmi jednoduchý způsob, jak změřit významnost
prvku v síti, v tomto případě autorů. Tato metoda je však příliš jednoduchá,
protože do výpočtu hodnoty centrality nezahrnuje uzly, které jsou od daného
uzlu vzdálenější než jeden skok. Tento fakt je známý problém a důvod pro
zavedení dalších a složitějších metod pro výpočet významnosti.

\subsection{Eigenvector}
Eigenvector centrality je míra vlivu uzlu v síti. Oproti jednoduché degree
centrality, která použila pouze počet přímých spojení, tahle míra navíc používá
významnost uzlů v těchto přímých spojeních. Pokud máme dva uzly se stejným
počtem přímých spojení, hodnota degree centrality bude u obou stejná. Pokud je
ale první uzel spojen s významnějšími uzly než druhý uzel, v metodě typu
eigenvector centrality bude první uzel hodnocen výše. Eigenvector centralitu
vypočteme pomocí numerických metod pro vlastní čísla, které vyřeší rovnici 
\begin{align}
{\bf Ax} &= \lambda {\bf x}
\end{align}

kde ${\bf A}$ je matice sousednosti (adjacency matrix), ${\bf x}$ je vlastní
vektor matice $A$ a je řešením této rovnice, která má více řešení. Ke každému
řešení náleží vlastní číslo $\lambda$. Podle Perron-Frobeniovy věty má pouze
jedno řešení všechny prvky vlastního vektoru ${\bf x}$ nezáporné
\cite{langvillemeyer}. Tohle řešení odpovídá největšímu vlastnímu číslu
$\lambda$ a jako jediné nás v problému eigenvector centrality zajímá. 

Pro výpočet použijeme iterační metodu
\begin{align}
x_u &=  \frac{1}{\lambda} \displaystyle\sum_{v \in G} {\bf A}_{uv} x_v \\
\end{align}
kde $x_u$ je prvek vlastního vektoru ${\bf x}$, který odpovídá uzlu $u$. $G$ je
množina uzlů sítě a ${\bf A}_{uv}$ je prvek matice sousednosti na řádku $u$ a
sloupci $v$, který je $1$, pokud existuje přímé spojení mezi uzly $u$ a $v$,
jinak $0$.

V tomto vyjádření eigenvector centrality zjistíme, že se jedná o přímé
rozšíření degree centrality. Výsledek z předchozí iterace použijeme jako vstup
do další iterace. Výpočet iterujeme, dokud nedosáhneme požadované přesnosti
výsledku.
\subsubsection{PageRank}
\subsection{Closeness}
% TODO zmínit největší komponentu
\subsubsection{Algoritmus}
\subsubsection{Aproximace}
\subsection{Betweeness}
\subsubsection{Brandesův algoritmus}
\subsubsection{Aproximace}
\subsection{Radius}

\section{Ostatní používané míry významnosti autorů}
\subsection{H-index}
\subsection{Impact factor}

\section{Metody porovnaní}
\subsection{Spearmanův koeficient pořadové korelace}
\subsection{Pearsonův korelační koeficient}

\section{Ocenění významných autorů}
% TODO kecy co to ty ceny jsou a k čemu jsou a proč je tu uvádim.
\subsection{ACM A.M. Turing Award}
ACM A.M. Turing Award je ocenění ročně udělované skupinou ACM (Association for
Computing Machinery) jedincům vybraným pro kontribuce technického ducha do
výpočetního světa.
\cite{turingaward}.

Turingova cena je brána jako nejvyšší vyznamenání v informatice a je lidově
nazývána Nobelovou cenou pro informatiku \cite[p.~317]{dasgupta}.

\subsection{ACM SIGMOD Edgar F. Codd Innovations Award}
ACM SIGMOD Edgar F. Codd Innovations Award je ohodnocení životního díla
skupinou ACM SIGMOD (Special Interest Group on Management of Data)  za
inovativní a vysoce ceněné kontribuce k rozvoji, porozumění a použití
databázových systémů a databází \cite{sigmodinnovations}.

\subsection{ACM Fellows}
\uv{The ACM Fellows Program} byl založen v roce 1993, aby našel a ocenil
vynikající členy ACM za jejich dílo v informatice a informační vědě a pro
jejich významné kontribuce pro účel ACM. Členové ACM Fellows slouží jako
význační kolegové, ke kterým ACM a jejich členové vzhlížejí jako k autoritám v
době rozvoje informačních technologií \cite{acmfellows}.

\subsection{ISI Highly Cited highlighted}
ISI Highly Cited je databáze často citovaných autorů v článcích posledního
desetiletí, které byly vydány institutem ISI (Institute for Scientific
Information). Ten v dnešní době spadá pod agenturu Thomson Reuters, na jejíchž webových stránkách nalezneme seznam autorů ISI Highly Cited highlighted z let 2000 až 2008 napříč 21 vědeckými obory \cite{highlycited}.

\chapter{Implementace}
\section{Načtení vstupních dat}

\section{Vytvoření citačních sítí}
\subsection{Síť publikací}
\subsection{Síť autorů}

\section{Analýza struktury sítě}

\section{Reprezentace citační sítě}
\subsection{All-pair shortest path}
\subsection{Single source shortest path}

\section{Hledání nejkratších cest}

\section{Knihovna pro SNA}
\subsection{Radius}
\subsection{Degree}
\subsection{Eigenvector}
\subsection{Closeness}
\subsection{Betweeness}
% TODO normalizace - vydělení maximální možnou hodnotou
\subsection{H-index}

\subsection{Paralelní výpočty}

\section{Citační databáze}
\subsection{DBLP}
DBLP \cite{DBLP} je webová bibliografická databáze v oboru informatiky,
která k listopadu 2012 obsahovala více než 2,1 milionu publikací. Pro tuto
práci používáme verzi z roku 2004.

\subsubsection{Charakteristika}
% TODO tzn. v jakym roce je nejvíc publikací atd.

\subsection{CiteSeer}
CiteSeer (nyní CiteSeer$^X$) \cite{citeseer} je považován za první
automatizovaný systém shromažďování publikací a autonomní indexace citací v
nich obsažených. Publikace jsou zejména z oboru informatiky a informační vědy.
V dnešní době obsahuje přes dva miliony dokumentů s téměř dvěma miliony autorů
a čtyřiceti miliony citací. Zde používáme verzi z roku 2005.

\subsubsection{Charakteristika}


\chapter{Výsledky}
\section{Struktura sítě}
\subsection{DBLP}
\subsubsection{Rozdělení vah}
\subsection{CiteSeer}
\subsubsection{Rozdělení vah}

\section{Porovnání nejvýznamnějších autorů}
\section{Porovnání implementovaných metod}

\section{Žebříčky významných autorů}
\subsection{DBLP}
\subsubsection{H-index}
\begin{center}
\begin{tabular}{|l|l|c|c|c|c|c|}
\hline
& {\bf Autor} & {\bf H-index} & {\bf Turing} & {\bf Codd} & {\bf Fellows} & {\bf ISI} \\
\hline
1 & MICHAEL STONEBRAKER & 28& & $\bullet$ & $\bullet$ &         \\
\hline
2 & DAVID J. DEWITT & 24& & $\bullet$ & $\bullet$ &         \\
\hline
3 & JEFFREY D. ULLMAN & 24& & $\bullet$ & $\bullet$ & $\bullet$ \\
\hline
4 & PHILIP A. BERNSTEIN & 22& & $\bullet$ & $\bullet$ & $\bullet$ \\
\hline
5 & RAKESH AGRAWAL & 21& & $\bullet$ & $\bullet$ &         \\
\hline
6 & WON KIM & 21& & $\bullet$ & $\bullet$ &         \\
\hline
7 & YEHOSHUA SAGIV & 20& &         &         &         \\
\hline
8 & CATRIEL BEERI & 20& &         & $\bullet$ & $\bullet$ \\
\hline
9 & MICHAEL J. CAREY & 20& & $\bullet$ & $\bullet$ &         \\
\hline
10 & SERGE ABITEBOUL & 19& & $\bullet$ & $\bullet$ & $\bullet$ \\
\hline
11 & HECTOR GARCIA-MOLINA & 19& & $\bullet$ & $\bullet$ & $\bullet$ \\
\hline
12 & UMESHWAR DAYAL & 19& & $\bullet$ & $\bullet$ &         \\
\hline
13 & CHRISTOS FALOUTSOS & 19& &         & $\bullet$ & $\bullet$ \\
\hline
14 & NATHAN GOODMAN & 18& & $\bullet$ &         &         \\
\hline
15 & JIM GRAY & 18& &         & $\bullet$ &         \\
\hline
16 & JEFFREY F. NAUGHTON & 18& &         & $\bullet$ &         \\
\hline
17 & RAGHU RAMAKRISHNAN & 18& &         &         &         \\
\hline
18 & RONALD FAGIN & 18& & $\bullet$ & $\bullet$ & $\bullet$ \\
\hline
19 & JENNIFER WIDOM & 18& & $\bullet$ & $\bullet$ &         \\
\hline
20 & DAVID MAIER & 17& & $\bullet$ & $\bullet$ & $\bullet$ \\
\hline
21 & BRUCE G. LINDSAY & 17& &         & $\bullet$ &         \\
\hline
22 & SHAMKANT B. NAVATHE & 16& &         &         &         \\
\hline
23 & C. MOHAN & 16& & $\bullet$ & $\bullet$ &         \\
\hline
24 & HAMID PIRAHESH & 16& &         & $\bullet$ &         \\
\hline
25 & H. V. JAGADISH & 16& &         & $\bullet$ &         \\
\hline
\end{tabular}
\end{center}

\subsubsection{Nevážený indegree}
\begin{center}
\begin{tabular}{|l|l|c|c|c|c|c|}
\hline
& {\bf Autor} & {\bf indegree} & {\bf Turing} & {\bf Codd} & {\bf Fellows} & {\bf ISI} \\
\hline
1  & MICHAEL STONEBRAKER & 1909&         &$\bullet$&$\bullet$& \\
\hline
2  & DAVID J. DEWITT & 1484&         &$\bullet$&$\bullet$& \\
\hline
3  & JIM GRAY & 1400&         &$\bullet$&$\bullet$& \\
\hline
4  & RAYMOND A. LORIE & 1276&         &         &$\bullet$& \\
\hline
5  & JEFFREY D. ULLMAN & 1180&         &$\bullet$&$\bullet$&$\bullet$ \\
\hline
6  & WON KIM & 1146&         &         &$\bullet$& \\
\hline
7  & PHILIP A. BERNSTEIN & 1145&         &$\bullet$&$\bullet$&$\bullet$ \\
\hline
8  & E. F. CODD & 1110&$\bullet$&         &$\bullet$& \\
\hline
9  & MICHAEL J. CAREY & 1110&         &$\bullet$&$\bullet$& \\
\hline
10 & UMESHWAR DAYAL & 1076&         &$\bullet$&$\bullet$& \\
\hline
11 & HECTOR GARCIA-MOLINA & 1020&         &$\bullet$&$\bullet$&$\bullet$ \\
\hline
12 & DAVID MAIER & 1017&         &$\bullet$&$\bullet$&$\bullet$ \\
\hline
13 & DONALD D. CHAMBERLIN & 966&         &$\bullet$&$\bullet$& \\
\hline
14 & RAKESH AGRAWAL & 907&         &$\bullet$&$\bullet$& \\
\hline
15 & PETER P. CHEN & 906&         &         &$\bullet$& \\
\hline
16 & SERGE ABITEBOUL & 848&         &$\bullet$&$\bullet$&$\bullet$ \\
\hline
17 & KAPALI P. ESWARAN & 847&         &         &         & \\
\hline
18 & MORTON M. ASTRAHAN & 846&         &         &         & \\
\hline
19 & FRANCCEDILOIS BANCILHON & 840&         &         &         & \\
\hline
20 & NATHAN GOODMAN & 819&         &         &         &$\bullet$ \\
\hline
21 & BRUCE G. LINDSAY & 806&         &         &$\bullet$& \\
\hline
22 & HAMID PIRAHESH & 803&         &         &$\bullet$& \\
\hline
23 & IRVING L. TRAIGER & 785&         &         &$\bullet$& \\
\hline
24 & EUGENE WONG & 762&         &         &         & \\
\hline
25 & JEFFREY F. NAUGHTON & 729&         &         &$\bullet$& \\
\hline
\end{tabular}
\end{center}

\subsubsection{Nevážený outdegree}
\begin{center}
\begin{tabular}{|l|l|c|c|c|c|c|}
\hline
& {\bf Autor} & {\bf indegree} & {\bf Turing} & {\bf Codd} & {\bf Fellows} & {\bf ISI} \\
\hline
1  & GERHARD WEIKUM & 872&&         &$\bullet$&         \\
\hline
2  & HECTOR GARCIA-MOLINA & 856&&$\bullet$&$\bullet$&$\bullet$\\
\hline
3  & RAKESH AGRAWAL & 761&&$\bullet$&$\bullet$&         \\
\hline
4  & MICHAEL J. CAREY & 758&&$\bullet$&$\bullet$&         \\
\hline
5  & DAVID J. DEWITT & 758&&$\bullet$&$\bullet$&         \\
\hline
6  & H. V. JAGADISH & 717&&         &$\bullet$&         \\
\hline
7  & MICHAEL STONEBRAKER & 677&&$\bullet$&$\bullet$&         \\
\hline
8  & RAGHU RAMAKRISHNAN & 652&&         &$\bullet$&         \\
\hline
9  & YANNIS E. IOANNIDIS & 649&&         &$\bullet$&         \\
\hline
10 & ABRAHAM SILBERSCHATZ & 636&&         &$\bullet$&         \\
\hline
11 & ELISA BERTINO & 635&&         &$\bullet$&         \\
\hline
12 & SHAMKANT B. NAVATHE & 629&&         &         &         \\
\hline
13 & PHILIP S. YU & 622&&         &$\bullet$&$\bullet$\\
\hline
14 & STEFANO CERI & 611&&         &         &         \\
\hline
15 & CHRISTOS FALOUTSOS & 607&&         &$\bullet$&         \\
\hline
16 & MATTHIAS JARKE & 586&&         &         &         \\
\hline
17 & GULTEKIN OUMLZSOYOGLU & 582&&         &         &         \\
\hline
18 & SERGE ABITEBOUL & 575&&$\bullet$&$\bullet$&$\bullet$\\
\hline
19 & NICK ROUSSOPOULOS & 568&&         &$\bullet$&         \\
\hline
20 & MIRON LIVNY & 559&&         &         &         \\
\hline
21 & STANLEY Y. W. SU & 558&&         &         &         \\
\hline
22 & HANS-JOUMLRG SCHEK & 557&&         &$\bullet$&         \\
\hline
23 & PATRICK VALDURIEZ & 547&&         &$\bullet$&         \\
\hline
24 & GOETZ GRAEFE & 546&&         &         &         \\
\hline
25 & CLEMENT T. YU & 542&&         &         &         \\
\hline
\end{tabular}
\end{center}

\subsubsection{Vážený indegree}
\begin{center}
\begin{tabular}{|l|l|c|c|c|c|c|}
\hline
& {\bf Autor} & {\bf indegree} & {\bf Turing} & {\bf Codd} & {\bf Fellows} & {\bf ISI} \\
\hline
1  & MICHAEL STONEBRAKER & 5946    &         &$\bullet$&$\bullet$&         \\
\hline
2  & DAVID J. DEWITT & 5733        &         &$\bullet$&$\bullet$&         \\
\hline
3  & JEFFREY D. ULLMAN & 4429      &         &$\bullet$&$\bullet$&$\bullet$\\
\hline
4  & JIM GRAY & 3982               &         &$\bullet$&$\bullet$&         \\
\hline
5  & MICHAEL J. CAREY & 3583       &         &$\bullet$&$\bullet$&         \\
\hline
6  & RAYMOND A. LORIE & 3501       &         &         &$\bullet$&         \\
\hline
7  & HECTOR GARCIA-MOLINA & 3275   &         &$\bullet$&$\bullet$&$\bullet$\\
\hline
8  & PHILIP A. BERNSTEIN & 3225    &         &$\bullet$&$\bullet$&$\bullet$\\
\hline
9  & SERGE ABITEBOUL & 3177        &         &$\bullet$&$\bullet$&$\bullet$\\
\hline
10 & RAKESH AGRAWAL & 3152         &         &$\bullet$&$\bullet$&         \\
\hline
11 & WON KIM & 2993                &         &         &$\bullet$&         \\
\hline
12 & DAVID MAIER & 2772            &         &$\bullet$&$\bullet$&$\bullet$\\
\hline
13 & E. F. CODD & 2736             &$\bullet$&         &$\bullet$&         \\
\hline
14 & YEHOSHUA SAGIV & 2575         &         &         &         &         \\
\hline
15 & CATRIEL BEERI & 2491          &         &         &$\bullet$&$\bullet$\\
\hline
16 & UMESHWAR DAYAL & 2465         &         &$\bullet$&$\bullet$&         \\
\hline
17 & RAGHU RAMAKRISHNAN & 2426     &         &         &$\bullet$&         \\
\hline
18 & CHRISTOS FALOUTSOS & 2413     &         &         &$\bullet$&         \\
\hline
19 & JENNIFER WIDOM & 2354         &         &$\bullet$&$\bullet$&         \\
\hline
20 & DONALD D. CHAMBERLIN & 2269   &         &$\bullet$&$\bullet$&         \\
\hline
21 & FRANCCEDILOIS BANCILHON & 2264&         &         &         &         \\
\hline
22 & JEFFREY F. NAUGHTON & 2186    &         &         &$\bullet$&         \\
\hline
23 & NATHAN GOODMAN & 2176         &         &         &         &$\bullet$\\
\hline
24 & HAMID PIRAHESH & 2135         &         &         &$\bullet$&         \\
\hline
25 & BRUCE G. LINDSAY & 2013       &         &         &$\bullet$&         \\
\hline
\end{tabular}
\end{center}

\subsubsection{Vážený outdegree}
\begin{center}
\begin{tabular}{|l|l|c|c|c|c|c|}
\hline
& {\bf Autor} & {\bf outdegree} & {\bf Turing} & {\bf Codd} & {\bf Fellows} & {\bf ISI} \\
\hline
1  & MICHAEL J. CAREY & 3239    &&$\bullet$&$\bullet$&         \\
\hline
2  & GERHARD WEIKUM & 3071      &&         &$\bullet$&         \\
\hline
3  & DAVID J. DEWITT & 2818     &&$\bullet$&$\bullet$&         \\
\hline
4  & PHILIP S. YU & 2614        &&         &$\bullet$&$\bullet$\\
\hline
5  & HECTOR GARCIA-MOLINA & 2512&&$\bullet$&$\bullet$&$\bullet$\\
\hline
6  & MICHAEL STONEBRAKER & 2316 &&$\bullet$&$\bullet$&         \\
\hline
7  & SERGE ABITEBOUL & 2297     &&$\bullet$&$\bullet$&$\bullet$\\
\hline
8  & H. V. JAGADISH & 2263      &&         &$\bullet$&         \\
\hline
9  & RAKESH AGRAWAL & 2240      &&$\bullet$&$\bullet$&         \\
\hline
10 & RAGHU RAMAKRISHNAN & 2059  &&         &$\bullet$&         \\
\hline
11 & CHRISTOS FALOUTSOS & 2042  &&         &$\bullet$&         \\
\hline
12 & WON KIM & 1902             &&         &$\bullet$&         \\
\hline
13 & ABRAHAM SILBERSCHATZ & 1867&&         &$\bullet$&         \\
\hline
14 & MIRON LIVNY & 1806         &&         &         &         \\
\hline
15 & GOETZ GRAEFE & 1789        &&         &         &         \\
\hline
16 & STEFANO CERI & 1775        &&         &         &         \\
\hline
17 & YANNIS E. IOANNIDIS & 1775 &&         &$\bullet$&         \\
\hline
18 & RICHARD HULL & 1692        &&         &$\bullet$&         \\
\hline
19 & HAMID PIRAHESH & 1685      &&         &$\bullet$&         \\
\hline
20 & HANS-JOUMLRG SCHEK & 1661  &&         &$\bullet$&         \\
\hline
21 & STANLEY Y. W. SU & 1651    &&         &         &         \\
\hline
22 & CLEMENT T. YU & 1630       &&         &         &         \\
\hline
23 & JEFFREY F. NAUGHTON & 1587 &&         &$\bullet$&         \\
\hline
24 & RICHARD T. SNODGRASS & 1558&&         &$\bullet$&         \\
\hline
25 & SHAMKANT B. NAVATHE & 1538 &&         &         &         \\
\hline
\end{tabular}
\end{center}


\subsubsection{PageRank}
Hodnoty PageRanku dosahují hodnot mezi $0$ a $1$. Pro účely přehlednosti byly v
této tabulce normalizovány na interval $0$ až $|V|$, tedy počet uzlů sítě.

\begin{center}
\begin{tabular}{|l|l|c|c|c|c|c|}
\hline
& {\bf Autor} & {\bf PageRank} & {\bf Turing} & {\bf Codd} & {\bf Fellows} & {\bf ISI} \\
\hline
1 & E. F. CODD & 179.324            & $\bullet$ &           & $\bullet$ &           \\
\hline
2 & MICHAEL STONEBRAKER & 137.371   &           & $\bullet$ & $\bullet$ &           \\
\hline
3 & JIM GRAY & 115.364              &           & $\bullet$ & $\bullet$ &           \\
\hline
4 & DONALD D. CHAMBERLIN & 114.010  &           &           & $\bullet$ &           \\
\hline
5 & RAYMOND A. LORIE & 107.204      &           &           & $\bullet$ &           \\
\hline
6 & PHILIP A. BERNSTEIN & 99.575    &           & $\bullet$ & $\bullet$ & $\bullet$ \\
\hline
7 & MORTON M. ASTRAHAN & 87.673     &           &           &           &           \\
\hline
8 & KAPALI P. ESWARAN & 87.167      &           &           &           &           \\
\hline
9 & PETER P. CHEN & 84.098          &           &           & $\bullet$ &           \\
\hline
10 & IRVING L. TRAIGER & 79.313     &           &           & $\bullet$ &           \\
\hline
11 & JOHN MILES SMITH & 78.833      &           &           &           &           \\
\hline
12 & JEFFREY D. ULLMAN & 74.323     &           & $\bullet$ & $\bullet$ & $\bullet$ \\
\hline
13 & EUGENE WONG & 68.319           &           &           &           &           \\
\hline
14 & DAVID J. DEWITT & 67.701       &           & $\bullet$ & $\bullet$ &           \\
\hline
15 & MIKE W. BLASGEN & 62.185       &           &           & $\bullet$ &           \\
\hline
16 & GIANFRANCO R. PUTZOLU & 61.585 &           &           &           &           \\
\hline
17 & BRADFORD W. WADE & 60.731      &           &           &           &           \\
\hline
18 & RUDOLF BAYER & 60.706          &           & $\bullet$ &           &           \\
\hline
19 & JAMES W. MEHL & 58.499         &           &           &           &           \\
\hline
20 & PATRICIA P. GRIFFITHS & 58.215 &           &           &           &           \\
\hline
21 & WON KIM & 57.946               &           & $\bullet$ & $\bullet$ &           \\
\hline
22 & W. FRANK KING III & 57.169     &           &           &           &           \\
\hline
23 & NATHAN GOODMAN & 56.791        &           &           &           & $\bullet$ \\
\hline
24 & PAUL R. MCJONES & 55.967       &           &           & $\bullet$ &           \\
\hline
25 & RONALD FAGIN & 54.766          &           & $\bullet$ & $\bullet$ & $\bullet$ \\
\hline
\end{tabular}
\end{center}

\subsubsection{Nevážený closeness}
\begin{center}
\begin{tabular}{|l|l|c|c|c|c|c|}
\hline
& {\bf Autor} & {\bf Closeness} & {\bf Turing} & {\bf Codd} & {\bf Fellows} & {\bf ISI} \\
\hline
1  & MICHAEL STONEBRAKER & 0.593& & $\bullet$ & $\bullet$ &           \\
\hline
2  & JIM GRAY & 0.560&           & $\bullet$ & $\bullet$ &           \\
\hline
3  & DAVID J. DEWITT & 0.556&           & $\bullet$ & $\bullet$ &           \\
\hline
4  & RAYMOND A. LORIE & 0.556&           &           & $\bullet$ &           \\
\hline
5  & JEFFREY D. ULLMAN & 0.546&           & $\bullet$ & $\bullet$ & $\bullet$ \\
\hline
6  & PHILIP A. BERNSTEIN & 0.546&& $\bullet$ & $\bullet$ & $\bullet$ \\
\hline
7  & E. F. CODD & 0.543& $\bullet$ &           & $\bullet$ &           \\
\hline
8  & DONALD D. CHAMBERLIN & 0.539& & & $\bullet$ &           \\
\hline
9  & WON KIM & 0.537&           & $\bullet$ & $\bullet$ &           \\
\hline
10 & UMESHWAR DAYAL & 0.535&         &$\bullet$&$\bullet$&         \\
\hline
11 & MICHAEL J. CAREY & 0.532&&$\bullet$&$\bullet$&         \\
\hline
12 & MORTON M. ASTRAHAN & 0.531&     &     &     &           \\
\hline
13 & DAVID MAIER & 0.529&         &$\bullet$&$\bullet$&$\bullet$\\
\hline
14 & KAPALI P. ESWARAN & 0.529 &    &   &     &    \\
\hline
15 & NATHAN GOODMAN & 0.527 & & $\bullet$ &         &         \\
\hline
16 & EUGENE WONG & 0.526& &  &   & \\
\hline
17 & IRVING L. TRAIGER & 0.525&           &           & $\bullet$ &           \\
\hline
18 & HECTOR GARCIA-MOLINA & 0.523 &&$\bullet$&$\bullet$&$\bullet$\\
\hline
19 & FRANCCEDILOIS BANCILHON & 0.520 &         &         &         &         \\
\hline
20 & BRUCE G. LINDSAY & 0.519&         &         &$\bullet$&         \\
\hline
21 & PETER P. CHEN & 0.518&           &           & $\bullet$ &           \\
\hline
22 & RAKESH AGRAWAL & 0.518&&$\bullet$&$\bullet$&         \\
\hline
23 & RONALD FAGIN & 0.517&           & $\bullet$ & $\bullet$ & $\bullet$ \\
\hline
24 & CATRIEL BEERI & 0.517&         &         &$\bullet$&$\bullet$\\
\hline
25 & THOMAS G. PRICE & 0.514 & & & & \\
\hline
\end{tabular}
\end{center}

\subsubsection{Vážený closeness}
\begin{center}
\begin{tabular}{|l|l|c|c|c|c|c|}
\hline
& {\bf Autor} & {\bf Closeness} & {\bf Turing} & {\bf Codd} & {\bf Fellows} & {\bf ISI} \\
\hline
1  & MICHAEL STONEBRAKER & 2.055& & $\bullet$ & $\bullet$ &           \\
\hline
2  & JIM GRAY & 2.047&           & $\bullet$ & $\bullet$ &           \\
\hline
3  & E. F. CODD & 2.043& $\bullet$ &           & $\bullet$ &           \\
\hline
4  & DAVID J. DEWITT & 2.017&           & $\bullet$ & $\bullet$ &           \\
\hline
5  & JEFFREY D. ULLMAN & 2.014&           & $\bullet$ & $\bullet$ & $\bullet$ \\
\hline
6  & RAYMOND A. LORIE & 2.012&           &           & $\bullet$ &           \\
\hline
7  & PHILIP A. BERNSTEIN & 2.011&& $\bullet$ & $\bullet$ & $\bullet$ \\
\hline
8  & MICHAEL J. CAREY & 1.993&&$\bullet$&$\bullet$&         \\
\hline
9  & DAVID MAIER & 1.979&         &$\bullet$&$\bullet$&$\bullet$\\
\hline
10 & EUGENE WONG & 1.973& &  &   & \\
\hline
11 & DONALD D. CHAMBERLIN & 1.969& & & $\bullet$ &           \\
\hline
12 & LAWRENCE A. ROWE & 1.967 &&$\bullet$&&\\
\hline
13 & NATHAN GOODMAN & 1.966 & & $\bullet$ &         &         \\
\hline
14 & YEHOSHUA SAGIV & 1.966&         &         &         &         \\
\hline
15 & HECTOR GARCIA-MOLINA & 1.960 &&$\bullet$&$\bullet$&$\bullet$\\
\hline
16 & IRVING L. TRAIGER & 1.955&           &           & $\bullet$ &           \\
\hline
17 & CATRIEL BEERI & 1.949&         &         &$\bullet$&$\bullet$\\
\hline
18 & BRUCE G. LINDSAY & 1.944&         &         &$\bullet$&         \\
\hline
19 & MORTON M. ASTRAHAN & 1.943&     &     &     &           \\
\hline
20 & JEFFREY F. NAUGHTON & 1.942&&         &$\bullet$&         \\
\hline
21 & JENNIFER WIDOM & 1.937&         &$\bullet$&$\bullet$&         \\
\hline
22 & RAGHU RAMAKRISHNAN & 1.936 &&         &$\bullet$&         \\
\hline
23 & MIRON LIVNY & 1.936&&         &         &         \\
\hline
24 & RANDY H. KATZ & 1.934 &&$\bullet$&&\\
\hline
25 & RAKESH AGRAWAL & 1.933&&$\bullet$&$\bullet$&         \\
\hline
\end{tabular}
\end{center}

\subsubsection{Navážený betweeness}
\begin{center}
\begin{tabular}{|l|l|c|c|c|c|c|}
\hline
& {\bf Autor} & {\bf Betweeness} & {\bf Turing} & {\bf Codd} & {\bf Fellows} & {\bf ISI} \\
\hline
1 & PHILIP A. BERNSTEIN & 62655703.293 & & $\bullet$ & $\bullet$ & $\bullet$ \\
\hline
2 & MICHAEL STONEBRAKER & 61738362.921 & & $\bullet$ & $\bullet$ &           \\
\hline
3 & DAVID J. DEWITT & 60335509.092     & & $\bullet$ & $\bullet$ &           \\
\hline
4 & JIM GRAY & 58452724.132            & & $\bullet$ & $\bullet$ &           \\
\hline
5 & UMESHWAR DAYAL & 58105048.655      & & $\bullet$ & $\bullet$ &         \\
\hline
6 & RAYMOND A. LORIE & 57606842.228    & &           & $\bullet$ &           \\
\hline
7 & DONALD D. CHAMBERLIN & 57435250.431& &           & $\bullet$ &           \\
\hline
8 & MICHAEL J. CAREY & 56191915.811    & & $\bullet$ & $\bullet$ &         \\
\hline
9 & JEFFREY D. ULLMAN & 56098986.122   & & $\bullet$ & $\bullet$ & $\bullet$ \\
\hline
10 & KAPALI P. ESWARAN & 55953909.624  & &           &           &           \\
\hline
11 & E. F. CODD & 55595773.178         & $\bullet$ & & $\bullet$ &           \\
\hline
12 & WON KIM & 55485910.707            & & $\bullet$ & $\bullet$ &           \\
\hline
13 & MORTON M. ASTRAHAN & 53967137.730 &         &         &         & \\
\hline
14 & DAVID MAIER & 53884993.441        & &$\bullet$&$\bullet$&$\bullet$\\
\hline
15 & FRANCCEDILOIS BANCILHON & 52436978.786 &  &  & &         \\
\hline
16 & NATHAN GOODMAN & 51776071.388& & $\bullet$ &         &         \\
\hline
17 & EUGENE WONG & 50457002.386&         &         &         & \\
\hline
18 & IRVING L. TRAIGER & 50067735.663&         &         &$\bullet$& \\
\hline
19 & HECTOR GARCIA-MOLINA & 49279794.248& & $\bullet$ & $\bullet$ & $\bullet$ \\
\hline
20 & CATRIEL BEERI & 49031169.516&         &         &$\bullet$&$\bullet$\\
\hline
21 & RONALD FAGIN & 48476621.189& & $\bullet$ & $\bullet$ & $\bullet$ \\
\hline
22 & BRUCE G. LINDSAY & 47956637.448&         &         &$\bullet$&         \\
\hline
23 & SERGE ABITEBOUL & 47196023.670& & $\bullet$ & $\bullet$ & $\bullet$ \\
\hline
24 & RAKESH AGRAWAL & 46621125.945&&$\bullet$&$\bullet$&         \\
\hline
25 & PATRICIA G. SELINGER & 45312957.343 & & $\bullet$ & $\bullet$ & \\ 
\hline
\end{tabular}
\end{center}


\subsubsection{Vážený betweeness}
\begin{center}
\begin{tabular}{|l|l|c|c|c|c|c|}
\hline
& {\bf Autor} & {\bf Betweeness} & {\bf Turing} & {\bf Codd} & {\bf Fellows} & {\bf ISI} \\
\hline
1  & MICHAEL STONEBRAKER & 51920270.604 & & $\bullet$ & $\bullet$ &   \\
\hline
2  & DAVID J. DEWITT & 47407633.796& & $\bullet$ & $\bullet$ &           \\
\hline
3  & JIM GRAY & 46202513.744& & $\bullet$ & $\bullet$ &           \\
\hline
4  & JEFFREY D. ULLMAN & 43255880.998  & & $\bullet$ & $\bullet$ & $\bullet$ \\
\hline
5  & MICHAEL J. CAREY & 40700171.932 & & $\bullet$ & $\bullet$ &         \\
\hline
6  & RAYMOND A. LORIE & 37350006.114 & &           & $\bullet$ &           \\
\hline
7  & PHILIP A. BERNSTEIN & 36361815.206 & & $\bullet$ & $\bullet$ & $\bullet$ \\
\hline
8  & LAWRENCE A. ROWE & 35035718.542 &&$\bullet$&&\\
\hline
9  & EUGENE WONG & 34499259.565&         &         &         & \\
\hline
10 & MIRON LIVNY & 34488016.860&&         &         &         \\
\hline
11 & YEHOSHUA SAGIV & 32954711.980& &         &         &         \\
\hline
12 & DONALD D. CHAMBERLIN & 32704585.298& &           & $\bullet$ &           \\
\hline
13 & C. MOHAN & 32417070.578& & $\bullet$ & $\bullet$ &         \\
\hline
14 & DAVID MAIER & 32132542.674& &$\bullet$&$\bullet$&$\bullet$\\
\hline
15 & NATHAN GOODMAN & 31719323.894& & $\bullet$ &         &         \\
\hline
16 & HECTOR GARCIA-MOLINA & 31080457.354& & $\bullet$ & $\bullet$ & $\bullet$ \\
\hline
17 & RANDY H. KATZ & 30977276.737 &&$\bullet$&&\\
\hline
18 & JENNIFER WIDOM & 30793150.109&         &$\bullet$&$\bullet$& \\
\hline
19 & RAKESH AGRAWAL & 30763284.150&&$\bullet$&$\bullet$&         \\
\hline
20 & E. F. CODD & 29634524.870& $\bullet$ & & $\bullet$ &           \\
\hline
21 & JEFFREY F. NAUGHTON & 29267110.900&&         &$\bullet$&         \\
\hline
22 & HAMID PIRAHESH & 29233790.884&&         &$\bullet$&         \\
\hline
23 & CATRIEL BEERI & 28857797.364&         &         &$\bullet$&$\bullet$\\
\hline
24 & BRUCE G. LINDSAY & 28566231.162&         &         &$\bullet$&         \\
\hline
25 & RAGHU RAMAKRISHNAN & 28057226.481 &&         &$\bullet$&         \\
\hline
\end{tabular}
\end{center}



\subsection{CiteSeer}

\subsubsection{PageRank}
\begin{center}
\begin{tabular}{|l|l|c|c|c|c|c|}
\hline
& {\bf Autor} & {\bf PageRank} & {\bf Turing} & {\bf Codd} & {\bf Fellows} & {\bf ISI} \\
\hline
1  & JOHN K. OUSTERHOUT & 413.275 &         &         &$\bullet$&         \\
2  & MARTIN E. HELLMAN & 336.552  &         &         &         &         \\
3  & WHITFIELD DIFFIE & 289.814   &         &         &         &         \\
4  & SENIOR MEMBER & 280.899      &         &         &         &         \\
5  & JACK J. DONGARRA & 279.157   &         &         &$\bullet$&$\bullet$\\
6  & VAN JACOBSON & 259.762       &         &         &         &         \\
7  & SCOTT SHENKER & 225.241      &         &         &$\bullet$&         \\
8  & S. KENT & 224.260            &         &         &         &         \\
9  & RANDAL E. BRYANT & 197.141   &         &         &$\bullet$&$\bullet$\\
10 & SALLY FLOYD & 196.267        &         &         &$\bullet$&$\bullet$\\
11 & LIXIA ZHANG & 194.574        &         &         &$\bullet$&         \\
12 & STUDENT MEMBER & 182.926     &         &         &         &         \\
13 & S. KIRKPATRICK & 181.416     &         &         &$\bullet$&$\bullet$\\
14 & C. D. GELATT & 181.416       &         &         &         &         \\
15 & M. P. VECCHI & 181.416       &         &         &         &         \\
16 & TAKEO KANADE & 178.883       &         &         &$\bullet$&         \\
17 & RANDOLPH BENTSON & 178.869   &         &         &         &         \\
18 & GEORGE W. FURNAS & 177.145   &         &         &$\bullet$&         \\
19 & RAKESH AGRAWAL & 175.358     &         &$\bullet$&$\bullet$&         \\
20 & DEBORAH ESTRIN & 173.030     &         &         &$\bullet$&         \\
21 & STEPHEN C. JOHNSON & 168.657 &         &         &         &         \\
22 & EDWARD H. ADELSON & 162.659  &         &         &         &$\bullet$\\
23 & KEN THOMPSON & 159.405       &$\bullet$&         &         &         \\
24 & ADI SHAMIR & 155.899         &$\bullet$&         &         &         \\
25 & MICHAEL J. KARELS & 153.567  &         &         &         &         \\
\hline
\end{tabular}
\end{center}

\subsubsection{Nevážený closeness}
\begin{center}
\begin{tabular}{|l|l|c|c|c|c|c|}
\hline
& {\bf Autor} & {\bf Closeness} & {\bf Turing} & {\bf Codd} & {\bf Fellows} & {\bf ISI} \\
\hline
1  & SENIOR MEMBER & 0.393          & &         &         &          \\
2  & JOHN K. OUSTERHOUT & 0.392     & &         &$\bullet$&          \\
3  & SCOTT SHENKER & 0.384          & &         &$\bullet$&          \\
4  & M. FRANS KAASHOEK & 0.383      & &         &$\bullet$&          \\
5  & STUDENT MEMBER & 0.380         & &         &         &          \\
6  & RAKESH AGRAWAL & 0.380         & &$\bullet$&$\bullet$&          \\
7  & HARI BALAKRISHNAN & 0.377      & &         &$\bullet$&          \\
8  & DEBORAH ESTRIN & 0.377         & &         &$\bullet$&          \\
9  & HECTOR GARCIA-MOLINA & 0.376   & &$\bullet$&$\bullet$&$\bullet$ \\
10 & FACHBEREICH INFORMATIK & 0.375 & &         &         &          \\
11 & VAN JACOBSON & 0.375           & &         &         &          \\
12 & RAJEEV MOTWANI & 0.374         & &         &$\bullet$&          \\
13 & SALLY FLOYD & 0.373            & &         &$\bullet$&$\bullet$ \\
14 & DAVID CULLER & 0.370           & &         &$\bullet$&          \\
15 & LIXIA ZHANG & 0.370            & &         &$\bullet$&          \\
16 & CHRISTOS FALOUTSOS & 0.370     & &         &$\bullet$&          \\
17 & IAN FOSTER & 0.370             & &         &$\bullet$&$\bullet$ \\
18 & STEVEN MCCANNE & 0.370         & &         &         &          \\
19 & PRABHAKAR RAGHAVAN & 0.369     & &         &$\bullet$&$\bullet$ \\
20 & JENNIFER WIDOM & 0.369         & &$\bullet$&$\bullet$&          \\
21 & ROBERT E. SCHAPIRE & 0.368     & &         &         &$\bullet$ \\
22 & ROBERT MORRIS & 0.368          & &         &         &          \\
23 & M. SATYANARAYANAN & 0.368      & &         &$\bullet$&          \\
24 & PETER B. DANZIG & 0.367        & &         &         &          \\
25 & VERN PAXSON & 0.367            & &         &$\bullet$&$\bullet$ \\
\hline
\end{tabular}
\end{center}

\subsubsection{Nevážený betweeness}
\begin{center}
\begin{tabular}{|l|l|c|c|c|c|c|}
\hline
& {\bf Autor} & {\bf Betweeness} & {\bf Turing} & {\bf Codd} & {\bf Fellows} & {\bf ISI} \\
\hline
1  & M. FRANS KAASHOEK & 10112159061.330 & &         &$\bullet$&         \\
2  & SCOTT SHENKER & 9785892051.377      & &         &$\bullet$&         \\
3  & SENIOR MEMBER & 8845140725.909      & &         &         &         \\
4  & VAN JACOBSON & 8813158813.753       & &         &         &         \\
5  & SALLY FLOYD & 8690842977.231        & &         &$\bullet$&$\bullet$\\
6  & LARRY L. PETERSON & 8630281410.114  & &         &$\bullet$&         \\
7  & HARI BALAKRISHNAN & 8544868651.846  & &         &$\bullet$&         \\
8  & JENNIFER WIDOM & 8512314665.858     & &$\bullet$&$\bullet$&         \\
9  & DEBORAH ESTRIN & 8414557973.610     & &         &$\bullet$&         \\
10 & MONICA S. LAM & 8394649393.784      & &         &$\bullet$&         \\
11 & LIXIA ZHANG & 8350916122.773        & &         &$\bullet$&         \\
12 & STEVEN MCCANNE & 8263572085.250     & &         &         &         \\
13 & M. SATYANARAYANAN & 8087193503.971  & &         &$\bullet$&         \\
14 & THOMAS E. ANDERSON & 8078380316.694 & &         &$\bullet$&         \\
15 & DON TOWSLEY & 8053219720.880        & &         &$\bullet$&$\bullet$\\
16 & JOHN K. OUSTERHOUT & 8039121635.247 & &         &$\bullet$&         \\
17 & PETER B. DANZIG & 7986359949.439    & &         &         &         \\
18 & SERGE ABITEBOUL & 7924035872.338    & &$\bullet$&$\bullet$&$\bullet$\\
19 & CHRISTOS FALOUTSOS & 7915737482.408 & &         &$\bullet$&         \\
20 & STUDENT MEMBER & 7854847262.677     & &         &         &         \\
21 & KEN KENNEDY & 7846470147.904        & &         &$\bullet$&$\bullet$\\
22 & Y H. KATZ & 7746528995.356          & &         &$\bullet$&         \\
23 & DAVID B. JOHNSON & 7657508456.644   & &         &         &         \\
24 & RAKESH AGRAWAL & 7615283090.821     & &$\bullet$&$\bullet$&         \\
25 & HUI ZHANG & 7588067468.401          & &         &$\bullet$&         \\
\hline
\end{tabular}
\end{center}


\begin{center}
\begin{tabular}{|l|l|c|c|c|c|c|}
\hline
& {\bf Autor} & {\bf H-index} & {\bf Turing} & {\bf Codd} & {\bf Fellows} & {\bf ISI} \\
\hline
1  & SCOTT SHENKER & 37        & &         &$\bullet$&         \\
2  & DEBORAH ESTRIN & 34       & &         &$\bullet$&         \\
3  & KEN KENNEDY & 33          & &         &$\bullet$&$\bullet$\\
4  & DOUGLAS C. SCHMIDT & 33   & &         &         &         \\
5  & DON TOWSLEY & 32          & &         &$\bullet$&$\bullet$\\
6  & HECTOR GARCIA-MOLINA & 31 & &$\bullet$&$\bullet$&$\bullet$\\
7  & THOMAS A. HENZINGER & 30  & &         &$\bullet$&$\bullet$\\
8  & RAKESH AGRAWAL & 29       & &$\bullet$&$\bullet$&         \\
9  & M. FRANS KAASHOEK & 29    & &         &$\bullet$&         \\
10 & JENNIFER WIDOM & 29       & &$\bullet$&$\bullet$&         \\
11 & WILLY ZWAENEPOEL & 28     & &         &$\bullet$&         \\
12 & HUI ZHANG & 27            & &         &$\bullet$&         \\
13 & IAN FOSTER & 27           & &         &$\bullet$&$\bullet$\\
14 & MONI NAOR & 27            & &         &         &         \\
15 & SALLY FLOYD & 26          & &         &$\bullet$&$\bullet$\\
16 & LUCA CARDELLI & 26        & &         &$\bullet$&         \\
17 & SERGE ABITEBOUL & 26      & &$\bullet$&$\bullet$&$\bullet$\\
18 & SENIOR MEMBER & 26        & &         &         &         \\
19 & BART SELMAN & 26          & &         &$\bullet$&         \\
20 & SEBASTIAN THRUN & 25      & &         &         &         \\
21 & OREN ETZIONI & 24         & &         &         &         \\
22 & DAVID J. DEWITT & 24      & &$\bullet$&$\bullet$&         \\
23 & DAPHNE KOLLER & 24        & &         &         &         \\
24 & RAJEEV ALUR & 24          & &         &$\bullet$&$\bullet$\\
25 & HARI BALAKRISHNAN & 24    & &         &$\bullet$&         \\
\hline
\end{tabular}
\end{center}



\chapter{Diskuse}
\section{Podobnost výsledků jednotlivých metod}
\section{Shoda výsledků s oceněními}
\section{Vliv vah na přesnost výsledků}
\section{Vstupní a výstupní hrany}


\chapter{Závěr}

\bibliographystyle{alpha}
\bibliography{references}
\end{document}
